\chapter{Introduction}

The promise of an optimising compiler is that the programmer can focus on writing maintainable code, entrusting the responsibility of \textit{optimising} said code to the compiler. 

Entrusting optimisation to the compiler is sufficient for many cases, but not all. For reasons that are both theoretical \citep{rice-53} and practical \citep{robinson-01}, compilers will never be able to identify \textit{all} opportunities for optimisation. What can be done when compiler optimisation is not enough?
% A more difficult question is thus: How does one write code that is both fast
% and maintainable \textbf{when one cannot trust the compiler to optimise one's code}?
% One approach involves providing explicit information about
% optimisation opportunities to the compiler, through language constructs.
% These often involve annotations about the program. For example, 
% in \texttt{JavaScript}, the \texttt{\#\_\_PURE\_\_} annotation allows the 
% programmer to identify pure functions. In \texttt{Haskell}, annotations can 
% be used to identify strict arguments.

Metaprogramming (for example, \texttt{C++} templates) \citep{abrahams-04} is one possible solution. In languages that support metaprogramming, programmers can identify what should be executed at compile-time. Importantly, programmers may write \textbf{maintainable} code, which when executed by the compiler, generates \textbf{efficient} code. For example, assume two implementations of the same (widely-used) function, one that runs faster on Arm, and one on Intel.

The maintainable way to switch between these functions is via a conditional:
\begin{ocaml}
let f () = if (cpu_type == Arm) then 
             (* return this program *)
           else 
             (* return this program *)
f ()
\end{ocaml}

However, it might be expensive to determine processor type at run-time. If efficiency is paramount, programmers might have to duplicate their code, doubling maintenance costs. With metaprogramming, however, we may write 
\begin{macocaml}
(* macro and $ can be read as annotations which 
   indicate that the code should be run at compile-time *)
macro m () = if (cpu_type == Arm) then 
               (* generate this program, to be run later *)
             else 
               (* gen this program, to be run later *)
$(m ())
\end{macocaml}
Importantly, the macro \texttt{m} is executed by the compiler, \textit{at compile time}. The Arm and Intel variants are \textit{generated from a single specification}, not manually duplicated.

Importantly, to support metaprogramming, languages have to provide the ability to build programs \textit{to be run later}. These ``suspended programs'' are best thought of as abstract syntax trees. The following code constructs the program $(\lambda x. x) 1$.
\begin{ocaml}
if (cpu_type == Arm) then 
  let build_app (f: (int -> int) expr) (v: int expr) = App(f, v)
  let id_ast: (int -> int) expr = Lam(Var(x), Var(x))
  build_app(id_ast, Int(1)) (* int expr *)
else 
  (*...*)
\end{ocaml}
% The Macocaml project aims to add type-safe, compile-time metaprogramming to 
% \texttt{OCaml}. A key aim of the project is to ensure that metaprogramming 
% interacts reliably, and well, with other \texttt{OCaml} language features. 

Effect handlers are a powerful language construct that can simulate many other language features (state, I/O, greenthreading) \citep{pretnar-15}, and have recently been added to \texttt{OCaml} \citep{sivaramakrishnan-21}.  It is thus strategic, and timely, to investigate the interaction between metaprogramming and effect handlers. 

Unfortunately, metaprogramming is known to interact unexpectedly, and poorly, with effect handlers, a problem known as \textit{scope extrusion} \citep{kiselyov-14}. Scope extrusion occurs when the programmer accidentally directs the compiler to generate code with unbound variables. In the following program, we store the variable \mintinline{ocaml}{Var(x)} into a heap cell (\mintinline{ocaml}{l}), and retrieve it outside its scope. The program thus evaluates to \mintinline{ocaml}{Var(x)}, which is unbound.

\begin{ocaml}
let l: int expr ref = new(Int(1)) in 
let id_ast : (int -> int) expr = Lam(Var(x), l := Var(x); Int(1)) in
!l
\end{ocaml} 

The problem of scope extrusion has been widely studied, resulting in multitudinous mechanisms for managing the interaction between metaprogramming and effect handlers. Some solutions adapt the type system (Refined Environment Classifiers \citep{kiselyov-16,isoda-24}, Closed Types \citep{calcagno-00}), others insert dynamic checks into the generated code \citep{kiselyov-14}. However, there are two issues.

First, we are a picky lot. Type-based approaches require unfeasible modification of the \texttt{OCaml} type-checker, and tend to limit expressiveness, disallowing a wide range of programs that do not lead to scope extrusion. Dynamic solutions are inefficient, reporting scope extrusion long after the error occurred, or unpredictable, allowing some programs, but disallowing morally equivalent programs. Further, many of the dynamic solutions have not been proved correct.

Second, and more importantly, we lacked a way to evaluate solutions fairly and holistically. Our evaluation criteria is three-pronged: correctness, efficiency, and expressiveness. Each solution is described in its own calculus, which made solutions difficult to compare. It is non-trivial to show that the definitions of scope extrusion in differing calculi agree. Hence, one solution may be correct with respect to its definition, but wrong with respect to another. Further, expressiveness and efficiency are inherently comparative criteria.

This thesis will solve both these problems, in decreasing order of priority. 

\section{Contributions}
Concretely, our contributions are:
\begin{enumerate}
    \item A novel ``universal'' calculus that allows for effect handlers both in the generating and generated code.
    \item In the calculus, three precise (but different) definitions of scope extrusion.  
    \item In the calculus, encodings of the following existing solutions: 
    \begin{enumerate} 
        \item Lazy dynamic check (with a proof of correctness)
        \item Eager dynamic check (with a refutation of correctness)
        \item Refined Environment Classifiers
    \end{enumerate}
    \item A new solution, a Best-Effort dynamic check, that we justify is correct, expressive, and efficient.
    \item Implementations of the three dynamic checks in Macocaml.
\end{enumerate}