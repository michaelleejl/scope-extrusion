\chapter{Introduction}

Many compilers perform a wide range of optimisations. Compiler optimisations empower programmers to focus on writing maintainable code, leaving the compiler to optimise said code. 

While compiler optimisation is often all one needs, it is difficult for compiler engineers to ensure, \textit{a priori}, that every user has all the optimisations they require \citep{rice-53, robinson-01}. When users discover that their desired optimisation has not been implemented, what can they do, beyond submitting a pull request, or optimising their code by hand?
% A more difficult question is thus: How does one write code that is both fast
% and maintainable \textbf{when one cannot trust the compiler to optimise one's code}?
% One approach involves providing explicit information about
% optimisation opportunities to the compiler, through language constructs.
% These often involve annotations about the program. For example, 
% in \texttt{JavaScript}, the \texttt{\#\_\_PURE\_\_} annotation allows the 
% programmer to identify pure functions. In \texttt{Haskell}, annotations can 
% be used to identify strict arguments.

Metaprogramming (for example, \texttt{C++} templates) \citep{abrahams-04} allows the user to ``teach the compiler a class of [new] tricks'' \citep{sheard-02}, and is therefore one possible solution. Metaprogramming allows programmers to write \textbf{maintainable} code which directs the compiler to generate \textbf{efficient} code. 

For example, assume a \texttt{memcpy} function whose performance is heavily system dependent. How would one write a function that picks, depending on the system, the most performant implementation?

One possibility is via a dynamic conditional:
\begin{ocaml}
let memcpy source target n = if (architecture == AArch64) then 
                               (* return this program *)
                             else 
                                match vector_extensions with
                                | None -> (* return this program *)
                                | Some(AVX512) -> (* return this program *)
                                | ...
memcpy source target n
\end{ocaml}

Since \mintinline{ocaml}{architecture} and \mintinline{ocaml}{vector_extensions} are known at compile-time, they may be eliminated by an optimising compiler. If the compiler does not perform this optimisation, programmers might have to duplicate their code, increasing maintenance costs. Alternatively, they may use pre-processing directives, though this may lead to awkward code structures. With metaprogramming, however, they may write: 
\begin{macocaml}
(* macro and $ can be read as annotations which 
   indicate that the code should be run at compile-time *)
macro memcpy source target n = if (architecture == AArch64) then 
                               (* generate this program, to be run later *)
                               else 
                                 match vector_extensions with
                                 | None -> (* generate this program *)
                                 | Some(AVX512) -> (* generate this program *)
                                 | ...
$(memcpy <<source>> <<target>> <<n>>)
\end{macocaml}
Importantly, the \texttt{memcpy} macro is executed by the compiler, guaranteeing that the branches will be eliminated. This process of using metaprogramming to select the most performant \texttt{memcpy} on any given system is used in practice, for instance, by the \texttt{xine} media player \citep{huceke-memcpy}.

To support metaprogramming, languages have to provide the ability to build programs \textit{to be run later}. These suspended programs are best thought of as abstract syntax trees (ASTs). The following code constructs the program $(\lambda x. x) 1$.
\begin{ocaml}
if (architecture == AArch64) then 
  let build_app (f: (int -> int) expr) (v: int expr) = App(f, v) in
  let id_ast: (int -> int) expr = Lam(Var(x), Var(x)) in
  build_app(id_ast, Int(1)) (* int expr *)
else 
  (*...*)
\end{ocaml}
% The MacoCaml project aims to add type-safe, compile-time metaprogramming to 
% \texttt{OCaml}. A key aim of the project is to ensure that metaprogramming 
% interacts reliably, and well, with other \texttt{OCaml} language features. 

Metaprogramming is known to interact unexpectedly with effects. This interaction can be extremely desirable, for example, allowing programmers to manually perform loop-invariant code-motion \citep{kiselyov-14} without making major modifications to their existing code \citep{lawall-94}. 

Unfortunately, this interaction can also be undesirable. One problem is \textit{scope extrusion} \citep{kiselyov-14}. Scope extrusion occurs when the programmer accidentally generates code with unbound variables. In the following program, effects and metaprogramming combine to extrude \mintinline{ocaml}{Var(x)} beyond its scope. The program stores \mintinline{ocaml}{Var(x)} in a heap cell (\mintinline{ocaml}{l}), and subsequently retrieves it outside its scope. The program evaluates to \mintinline{ocaml}{Var(x)}, which is unbound.

\begin{ocaml}
let l: int expr ref = new(Int(1)) in 
let id_ast : (int -> int) expr = Lam(Var(x), l := Var(x); Int(1)) in
!l
\end{ocaml} 

This example illustrates how the state effect could cause scope extrusion. The choice of effect is arbitrary. Scope extrusion could be caused by many other effects, like exceptions.

To parameterise over the specific effect, I turn to effect handlers. Effect handlers, which have recently been added to \texttt{OCaml} \citep{sivaramakrishnan-21}, are a powerful language construct that can simulate many effects (state, I/O, greenthreading) \citep{pretnar-15}. It is thus strategic, and timely, to study scope extrusion in the context of effect handlers. 

The problem of scope extrusion has been widely studied, resulting in multitudinous mechanisms for managing the interaction between metaprogramming and effects. Some solutions adapt the type system (Refined Environment Classifiers \citep{kiselyov-16,isoda-24}, Closed Types \citep{calcagno-00}), others insert dynamic checks into the generated code \citep{kiselyov-14}. However, there are two issues.

First, I am not studying scope extrusion abstractly. The MacoCaml \citep{xie-2022} project aims to extend the \texttt{OCaml} programming language, which has effect handlers, with compile-time metaprogramming. However, until now, the MacoCaml project has had no clear policy on how metaprogramming and effect handlers should interact. The context adds constraints: type-based approaches require unfeasible modification of the \texttt{OCaml} type-checker, and tend to limit expressiveness, disallowing a wide range of programs that do not lead to scope extrusion. Dynamic solutions are inefficient, reporting scope extrusion long after the error occurs, or unpredictable, allowing some programs, but disallowing morally equivalent programs. Further, many of the dynamic solutions have not been proved correct.

Second, and more importantly, I lacked a way to evaluate solutions fairly and holistically. My evaluation criterion is three-pronged: correctness, expressiveness, and efficiency. Each solution is described in its own calculus, which made solutions difficult to compare. It is non-trivial to show that the definitions of scope extrusion in differing calculi agree. Hence, one solution may be correct with respect to its own definition, but wrong with respect to another. Further, expressiveness and efficiency are inherently comparative criteria.

This thesis solves both these problems. In \Cref{chapter:calculus}, I design a novel, common language for encoding and evaluating different solutions. In \Cref{chapter:scope-extrusion}, I use the designed language to formally describe and evaluate various checks. This process led to a novel best-effort dynamic check (\Cref{section:best-effort-check}), which I argue lands in a goldilocks zone, and should be adopted by MacoCaml.

\section{Contributions}

Concretely, the contributions of this work are:
\begin{enumerate}
    \item A novel two-stage calculus, \calculusName{}, that allows for effect handlers at both stages: compile-time and run-time (\Cref{chapter:calculus}). \calculusName{} is well-typed, has the expected metatheoretic properties (\Cref{section:metatheory}), and is designed to facilitate comparative evaluation of different scope extrusion checks (\Cref{chapter:scope-extrusion}).
    \item In \calculusName{}, a formal description of the MetaOCaml check described by \citet{kiselyov-14} (\Cref{subsubsection:eager-dynamic-check}), as well as an evaluation of its correctness (\Cref{subsection:eager-dynamic-correctness}), expressiveness (\Cref{subsection:eager-dynamic-expressiveness}), and efficiency (\Cref{subsection:eager-dynamic-efficiency}).
    \item In \calculusName{}, a formal description of a novel dynamic best-effort check (\Cref{section:best-effort-check}), which I prove correct (\Cref{subsection:best-effort-correct}), and argue, by comparison with other checks, lands in a goldilocks zone between expressiveness (\Cref{subsection:best-effort-expressive}) and efficiency (\Cref{subsection:best-effort-efficient}). 
    \item An extension of \calculusName{} with refined environment classifiers \citep{kiselyov-16}, with a proof of correctness via logical relation (\Cref{subsection:rec-formal-correctness}), and an evaluation of its expressiveness compared to other checks (\Cref{subsection:rec-formal-expressiveness}).
    \item Implementations of the three dynamic checks in MacoCaml. The best-effort check has been merged into MacoCaml.
\end{enumerate}

% The scope of this dissertation thus extends to effect systems with \textbf{deep} (but \textbf{unnamed}) handlers and \textbf{multi-shot continuations}.