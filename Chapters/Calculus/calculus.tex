

\chapter{Calculus}
\begin{figure}[H]
  \centering 
  \begin{tikzpicture}
    \begin{scope}[every node/.style={inner sep=0.5cm}, font=\ttfamily\scriptsize, anchor = east, rounded corners, line width = 0.4mm]
  \node[align=left, fill=sourceBackground, draw=sourceHighlight] (source-program) {\,\$(\textbf{do}$\, x \leftarrow \equote[\return{0}]$\\
  \,\quad \textbf{in} $(\lambda z. \return{z})(x))$};
  \node[align=left, fill=coreBackground, draw=coreHighlight] (core-program) at ($(source-program.east) + (6.4cm, 0cm)$){\textbf{tls}(\textbf{do}$\, x \leftarrow \return{\texttt{Ret}(\texttt{Nat}(0))}$ \\ 
  \quad\quad\textbf{in} $(\lambda z. \return{z})(x)$$)$};

  \node[] (normal-form) at ($(core-program.east) + (4cm, 0cm)$) {Ret$($Nat$(0))$};
  \end{scope}

  \begin{scope}[every node/.style={font = \scriptsize}]
    \node[fill=sourceHighlight] (source-lang) at ($(source-program.south) + (0cm, 0cm)$ ) {\textcolor{white}\sourceLang{}};
    \node[fill=coreHighlight] (core-lang) at ($(core-program.south) + (0cm, 0cm)$ ) {\textcolor{white}\coreLang{}};
  \end{scope}

  \begin{scope}[scale=8]
  \node[] (elaboration) at ($(source-program.east) + (0.095cm, -0.004cm)$) {\Huge$\leadsto$};
  \node[] (execution) at ($(core-program.east) + (0.095cm, 0.005cm)$) {\Huge$\rightarrow^{\text{\normalsize{$*$}}}$};
  \end{scope}

  \begin{scope}[every node/.style={font = \sffamily\tiny, anchor = south}]
  \node[] (elaboration-caption) at ($(elaboration) + (0cm, 0.3cm)$) {\textbf{Elaboration}};
  \node[align=center] (execution-caption) at ($(execution) + (0cm, 0.3cm)$) {\textbf{Compile-Time} \\ \textbf{Execution}};
  \end{scope}

  \end{tikzpicture}
  
  \caption{\calculusName{} is first elaborated into \coreLang{}, which is then executed \textbf{at compile-time} to obtain the AST of a run-time program.}%
  \label{fig:elaboration-then-execution}
\end{figure}

To re-iterate, I am considering the interaction between homogenous, compile-time, two-stage \textbf{metaprogramming} (\Cpageref{section:metaprogramming-technical}), and an \textbf{effect system} with deep handlers and multi-shot continuations (\Cpageref{section:effects-technical}). In this chapter, I describe a calculus, \calculusName{}, for studying said interaction. \calculusName{} will have both metaprogramming and effect handlers. To the best of my knowledge, this is the first calculus in which one may write effectful compile-time code that generates effectful run-time code. However, \calculusName{} will not mediate the interaction between metaprogramming and effects: scope extrusion prevention is not a language feature. Rather, the aim will be to implement different scope extrusion checks as algorithms in \calculusName{}, such that the checks may be evaluated in a comparative fashion. 

Programs written in \calculusName{} cannot be directly executed. Rather, following the style of \citet{xie-2023}, one must first elaborate (or compile) from \calculusName{} (the ``source'' language) to a ``core'' language,\, \coreLang{}.\, Programs written in \coreLang{} may then be executed, to obtain the AST of a run-time program. This process is summarised in \Cref{fig:elaboration-then-execution}. Elaboration is necessary, since it is the point at which dynamic checks may be inserted. 

In this chapter, I will first introduce \sourceLang{} (\Cref{section:source-lang}), then \coreLang{} (\Cref{section:core-lang}). Following this, I will describe the elaboration from \sourceLang{} to \coreLang{} (\Cref{section:elaboration}). Finally, I will discuss the metatheoretic properties of \calculusName{} (\Cref{section:metatheory}). 
% At a high-level, however, the names are suggestive: \sourceLang{} extends \efflang{} with quotation $\equote$ and splicing $\splice$, and \coreLang{} extends \efflang{} with AST constructors. 

\section{The source language: \texorpdfstring{\sourceLang{}}{Lambda-Op-Quote-Splice}}\label{section:source-lang}
\sourceLang{} extends \efflang{} with quotes and splices. Recall that \efflang{}, following a fine-grained call-by-value approach, divides terms into two syntactic categories, values $v$ and computations $c$. \sourceLang{} is similar, dividing terms into values $v$ and expressions $e$. We add quote and splice as follows:

\[e ::= \ldots \mid \equote \mid \splice \]
Notice that we cannot quote values: we \textit{must} generate effectful programs.

\subsection{Type System}


\begin{figure}
  \begin{source-desc}
    $\begin{array}{lllr}
    {\textbf{\large {Effects Row}}}\\\\
    \textbf{Run-Time} & \xi ::= \cdot \mid \xi \cup \{ \textsf{op}_i^{0}\} \\
    \textbf{Compile-Time} & \Delta ::= \cdot \mid \Delta \cup \{ \textsf{op}_i^{-1}\} \\\\
    \end{array}$

\vspace{5mm}
  $\begin{array}{lllr}
    {\textbf{\large {Types}}}\\\\ \vspace{0.4mm}
    \textbf{Level 0} & \text{Values} & T^0 ::= \mathbb{N}^0 & \text{\footnotesize{naturals}} \\ \vspace{0.4mm}
    && \quad\quad\,\, \mid {(\functionType[\xi]{T_1^0}{T_2^{0}})}^{0} & \text{\footnotesize{functions}} \\ \vspace{0.4mm}
    && \quad\quad\,\, \mid {(\continuationType[\xi]{T_1^0}{T_2^{0}})}^{0} & \text{\footnotesize{continuations}} \\\vspace{0.4mm}
    & \text{Computations} & T^0 \, ! \, \xi \\\vspace{0.8mm}
     && \mid T^0 \, ! \,  \Delta;\xi & \\\vspace{0.8mm}
    & \text{Handlers} & (\handlerType{T_1^{0} \, ! \, \xi}{T_2^{0} \, ! \, \xi'})\, !\, \Delta\\ \\ 

    \textbf{Level $-$1} & \text{Values} & T^{-1} ::= \mathbb{N}^{-1} & \text{\footnotesize{naturals}} \\ \vspace{0.4mm}
    && \quad\quad\quad \mid {(\functionType{T_1^{-1}}{T_2^{-1}})}^{-1} & \text{\footnotesize{functions}} \\\vspace{0.4mm}
    && \quad\quad\quad \mid {(\continuationType{T_1^{-1}}{T_2^{-1}})}^{-1} & \text{\footnotesize{continuations}} \\\vspace{0.4mm}
    && \quad\quad\quad \mid {\textsf{Code}({T^{0} \, ! \, \xi})}^{-1} & \text{\footnotesize{run-time code}} \\\vspace{0.8mm}
    & \text{Computations} & T^{-1} \, ! \, \Delta \\\vspace{0.8mm}
    & \text{Handlers} & \handlerType{T_1^{-1} \, ! \, \Delta}{T_2^{-1} \, ! \, \Delta'}
  \end{array}$
  \end{source-desc}
  \caption{\sourceLang{} types. I highlight three important elements: first, stratifying types into two levels, $0$ and $-1$. Second, stratifying effects into two levels, $\xi$ (for run-time effects) and $\Delta$ for compile-time effects. Third, the \textsf{Code} type at level $-1$ allows for compile-time programs to manipulate ASTs of run-time code.}
  \label{fig:source-types}
\end{figure}

I will now introduce the \sourceLang{} type system, by first introducing the types, and then the typing rules. The \sourceLang{} types are summarised in \Cref{fig:source-types}. I highlight three important details: types are stratified into two levels ($-1$ for compile-time and $0$ for run-time), effect rows are similarly stratified, and run-time code is made available at compile-time via a \textsf{Code} type.

First, \textbf{types are stratified into two levels, $T^0$ (run-time), and $T^{-1}$ (compile-time).}

  To motivate this stratification, consider the following question: what is the type of the number $3$ in \sourceLang{}? Perhaps surprisingly, the answer is not $\mathbb{N}$. Since we are working with a two-stage system, we must be careful to disambiguate between run-time naturals and compile-time naturals, since these are not interchangeable. For example, the following program should not be well-typed, since $3$ is a compile-time natural, whereas $x$ is a run-time natural. 
  \[\lambda x: \mathbb{N}. \, \$(3 + x)\]
  However, removing the splice makes the program well-typed
  \[\lambda x: \mathbb{N}. \, 3 + x\]
  Following \citet{xie-2023}, I introduce integer levels to enforce separation between compile-time and run-time naturals. While the precise notion of level is slightly more involved (see below), for my purposes, it is sufficient to think of level $0$ as run-time (so $\mathbb{N}^0$ is a run-time natural), and $-1$ for compile-time. The ill-typed example becomes
  \[\lambda x: \mathbb{N}^0. \, \$(\textcolor{comment}{(\textcolor{black}{3}:\mathbb{N}^{-1})} + \textcolor{comment}{(\textcolor{black}{x}:\mathbb{N}^{0})})\]
  and the well-typed example 
  \[\lambda x: \mathbb{N}^0. \, \textcolor{comment}{(\textcolor{black}{3}:\mathbb{N}^{0})} + \textcolor{comment}{(\textcolor{black}{x}:\mathbb{N}^{0})}\]
  More precisely, levels are defined as follows:

  \begin{definition}[Level]{sourceHighlight}
    The level of an expression $e$ is calculated by subtracting the number of surrounding splices from the number of surrounding quotations.
  \end{definition}

  The definition of level generalises to multi-stage languages, where negative levels ($-1, -2, \ldots$) represent compile-time and non-negative levels ($0, 1, \ldots$) represent run-time. In a multi-staged language, separation is even more granular: for example, level $1$ and level $0$ naturals, despite both being run-time naturals, are disambiguated. However, since we only deal with two stages, we only ought to consider two levels, $0$ and $-1$. The definition further implies that the ``default'' level, in the absence of quotes and splices, is level $0$. Intuitively, in the absence of quotes and splices, the programmer is ignoring metaprogramming facilities, and constructing a run-time program. 


  Notice that the opening question was slightly devious\footnote{sorry}! We cannot assign a type to \textit{program fragments}, like $3$, since without knowledge of the wider context, we cannot know which level we are at: in the ill-typed example, $3$ occurs under a splice, but no quotes, so it has type $\mathbb{N}^{-1}$, and in the well-typed example, it has type $\mathbb{N}^0$. When giving program fragments, like $3$, I will always assume that the starting level is $0$. 

  Second, \textbf{effect rows are stratified into $\xi$ (run-time) and $\Delta$ (compile-time).}\\
  In the following example, we print $1$ at compile-time, and $2$ at run-time. Further, we read an integer at run-time.
\[\splice[(\textbf{\texttt{print}}(1); \equote[\textbf{\texttt{print}}(2); \textbf{\texttt{readInt}}()])]\]
Hence, $\Delta = \{ \textbf{\texttt{print}} \}$ and $\xi = \{ \textbf{\texttt{print}}, \textbf{\texttt{readInt}} \}$. We may now disambiguate between different computation types:
\begin{enumerate}[leftmargin=5.8\parindent]
  \item[$T^0 \quad\quad\,\,$] Compile-time value, run-time value (value types) \\
  \textit{Example}: The type of $x$ in $\splice[(\bind{x}{\equote[1]}{\return{x}})]$
  \item[$T^0 \, ! \, \xi \quad\;$] Compile-time value, run-time computation  \\
  \textit{Example}: The type of $x$ in $\splice[(\bind{x}{\equote[\return{1}]}{\return{x}})]$
  \item[$T^0 \, ! \, \Delta \quad$] Compile-time computation, run-time value \\
  \textit{Example}: $1$, $\lambda x. \return{x}$ 
  \item[$T^0 \, ! \, \Delta; \xi$] Compile and run-time computation \\
  \textit{Example}: $\$(\print{2}; \equote[\return{1}])$
\end{enumerate}
% As a result of the type system, we never encounter $T^0 \, ! \, \Delta$ (compile-time computations that are run-time values).

Third, \textbf{there is an level $-1$ \textsf{Code} type, representing run-time ASTs.}\\ 
  By stratifying types to two levels, we have ensured that run-time (resp.\ compile-time) terms only interact with run-time (compile-time) terms. However, to enable meta-programming, run-time terms, \textit{should be available} at compile-time as ASTs. This is exactly the role of the \textsf{Code} type, thus allowing level $-1$ programs to manipulate ASTs of level $0$ terms. 
% Note that if we had a multi-level system, then there would be a \texttt{Code} type at level $0$, representing the ASTs of level $1$ terms. The asymmetry arises from the restriction to two levels. 

Putting it all together, we can now interpret complex types, like 
\[(\functionType[\{ \texttt{get} \}]{{(\textsf{Code}(\mathbb{N}^0 ! \{ \texttt{print} \}))}^{-1}}{(\textsf{Code}(\mathbb{N}^0 \, ! \, \{ \texttt{print}, \texttt{readInt} \}))^{-1}}\]
Reading from the outside-in:
\begin{enumerate}
  \item This is a level $-1$, or compile-time function
  \item The function has suspended compile-time effect \texttt{get} 
  \item The function expects an AST of some run-time computation which evaluates to a run-time natural ($\mathbb{N}^0$) and has unhandled run-time effect \texttt{print}
  \item The function returns an AST of some run-time computation which evaluates to a run-time natural and has unhandled run-time effects \texttt{print} and \texttt{readInt}
\end{enumerate}
\subsubsection{Typing Judgement}
Having described the types, I now present the type system. The typing rules are collated in \Cref{fig:source-cq-typing-rules,fig:source-s-typing-rules}. I will first explain the typing judgement, then highlight some key rules, and finally present an example typing derivation. 

The shape of the typing judgement is mostly familiar, though, as in \citet{xie-2023}, I add level information and compiler modes. Level information will turn out to be redundant, but we will revisit this later.
\[\Gamma \vdash^{\text{Level}}_{\text{Mode}} e: T\]
% I shall explain these in turn. 

\textbf{Level}. Recall that, when describing the \efflang{} types, I argued that one cannot type a program fragment, like $3$, directly. One must also know the \textit{level} ($0$ or $-1$), which is accordingly attached to the typing judgement.


\newcommand{\compilemode}{\textbf{\textsf{\textcolor{compile}{c}}}}
\newcommand{\splicemode}{\textbf{\textsf{\textcolor{splice}{s}}}}
\newcommand{\quotemode}{\textbf{\textsf{\textcolor{quote}{q}}}}
\textbf{Mode}. For the purposes of elaboration, it can be useful to keep classify code into three categories:

\begin{enumerate}
  \item[\compilemode] Code that is \textcolor{compile}{\textbf{ambient}} and \textcolor{compile}{\textbf{inert}}.\\
  \textit{No surrounding quotes or splices}
  \item[\splicemode] Code that \textcolor{splice}{\textbf{manipulates ASTs}} at compile-time. \\
  \textit{Last surrounding annotation is a splice}
  \item[\quotemode] Code that \textcolor{quote}{\textbf{builds ASTs}} to be manipulated at compile time. \\
  \textit{Last surrounding annotation is a quote}
\end{enumerate}

To illustrate the purpose of the modes, consider the following meta-program, which evaluates to the AST of $\lambda x. 1+2+3$. This program adopts the shorthand of \Cref{subsection:effect-handler-calculus}, where \texttt{\textbf{return}}s are implicitly inserted (for example, $\equote[1]$ should really be $\equote[\return{1}]$), and we elide the order of operation ($1+2$ should really be written out using \textbf{\texttt{do}}). The emphasis is on the three modes:

\begin{center}
  \begin{tikzpicture}
  \node (example) {$\lambda x. \, \splice[(\bind{f}{(\lambda y. \equote[{\splice[(y)] + 2}])}{\bind{a}{\equote[1]}{f a}}]) +3$};
  \draw[fill = compile, draw=compile, anchor = north west] ($(example.south west) + (0.05cm, 0.06cm)$) rectangle ($(example.south west) + (0.7cm, -0cm)$);
  
  \draw[fill = splice, draw=splice, anchor = north west] ($(example.south west) + (1.2cm, 0.06cm)$) rectangle ($(example.south west) + (3.3cm, -0cm)$);
  
  \draw[fill = splice, draw=splice, anchor = north west] ($(example.south west) + (4.08cm, 0.06cm)$) rectangle ($(example.south west) + (4.28cm, -0cm)$);
  
  \draw[fill = quote, draw=quote, anchor = north west] ($(example.south west) + (4.58cm, 0.06cm)$) rectangle ($(example.south west) + (5.15cm, -0cm)$);
  
  \draw[fill = splice, draw=splice, anchor = north west] ($(example.south west) + (5.75cm, 0.06cm)$) rectangle ($(example.south west) + (7.55cm, -0cm)$);
  
  \draw[fill = quote, draw=quote, anchor = north west] ($(example.south west) + (7.98cm, 0.06cm)$) rectangle ($(example.south west) + (8.17cm, -0cm)$);
  
  \draw[fill = splice, draw=splice, anchor = north west] ($(example.south west) + (8.6cm, 0.06cm)$) rectangle ($(example.south west) + (9.55cm, -0cm)$);
  
  \draw[fill = compile, draw=compile, anchor = north west] ($(example.south west) + (9.8cm, 0.06cm)$) rectangle ($(example.south west) + (10.5cm, -0cm)$);
  
  \end{tikzpicture}
  \end{center}

\begin{enumerate}
  \item[\compilemode] Identifies AST nodes that are \textcolor{compile}{\textbf{ambient}} (within which computation may take place) and \textcolor{compile}{\textbf{inert}} (cannot themselves be manipulated at compile-time). 
  
  \begin{center}
    \begin{tikzpicture}
    \node (example) {$\lambda x. \, \splice[(\bind{f}{(\lambda y. \equote[{\splice[(y)] + 2}])}{\bind{a}{\equote[1]}{f a}}]) +3$};
    \draw[fill = compile, draw=compile, anchor = north west] ($(example.south west) + (0.05cm, 0.06cm)$) rectangle ($(example.south west) + (0.7cm, -0cm)$);
    
    \draw[fill = fade, draw=fade, anchor = north west] ($(example.south west) + (1.2cm, 0.06cm)$) rectangle ($(example.south west) + (3.3cm, -0cm)$);
    
    \draw[fill = fade, draw=fade, anchor = north west] ($(example.south west) + (4.08cm, 0.06cm)$) rectangle ($(example.south west) + (4.28cm, -0cm)$);
    
    \draw[fill = fade, draw=fade, anchor = north west] ($(example.south west) + (4.58cm, 0.06cm)$) rectangle ($(example.south west) + (5.15cm, -0cm)$);
    
    \draw[fill = fade, draw=fade, anchor = north west] ($(example.south west) + (5.75cm, 0.06cm)$) rectangle ($(example.south west) + (7.55cm, -0cm)$);
    
    \draw[fill = fade, draw=fade, anchor = north west] ($(example.south west) + (7.98cm, 0.06cm)$) rectangle ($(example.south west) + (8.17cm, -0cm)$);
    
    \draw[fill = fade, draw=fade, anchor = north west] ($(example.south west) + (8.6cm, 0.06cm)$) rectangle ($(example.south west) + (9.55cm, -0cm)$);
    
    \draw[fill = compile, draw=compile, anchor = north west] ($(example.south west) + (9.8cm, 0.06cm)$) rectangle ($(example.south west) + (10.5cm, -0cm)$);
    
    \end{tikzpicture}
    \end{center}

  \item[\splicemode] Identifies code that can be executed at compile-time to \textcolor{splice}{\textbf{manipulate ASTs}}. Will be fully reduced at compile-time, and will not appear at run-time. 
  \begin{center}
    \begin{tikzpicture}
    \node (example) {$\lambda x. \, \splice[(\bind{f}{(\lambda y. \equote[{\splice[(y)] + 2}])}{\bind{a}{\equote[1]}{f a}}]) +3$};
    \draw[fill = fade, draw=fade, anchor = north west] ($(example.south west) + (0.05cm, 0.06cm)$) rectangle ($(example.south west) + (0.7cm, -0cm)$);
    
    \draw[fill = splice, draw=splice, anchor = north west] ($(example.south west) + (1.2cm, 0.06cm)$) rectangle ($(example.south west) + (3.3cm, -0cm)$);
    
    \draw[fill = splice, draw=splice, anchor = north west] ($(example.south west) + (4.08cm, 0.06cm)$) rectangle ($(example.south west) + (4.28cm, -0cm)$);
    
    \draw[fill = fade, draw=fade, anchor = north west] ($(example.south west) + (4.58cm, 0.06cm)$) rectangle ($(example.south west) + (5.15cm, -0cm)$);
    
    \draw[fill = splice, draw=splice, anchor = north west] ($(example.south west) + (5.75cm, 0.06cm)$) rectangle ($(example.south west) + (7.55cm, -0cm)$);
    
    \draw[fill = fade, draw=fade, anchor = north west] ($(example.south west) + (7.98cm, 0.06cm)$) rectangle ($(example.south west) + (8.17cm, -0cm)$);
    
    \draw[fill = splice, draw=splice, anchor = north west] ($(example.south west) + (8.6cm, 0.06cm)$) rectangle ($(example.south west) + (9.55cm, -0cm)$);
    
    \draw[fill = fade, draw=fade, anchor = north west] ($(example.south west) + (9.8cm, 0.06cm)$) rectangle ($(example.south west) + (10.5cm, -0cm)$);
    
    \end{tikzpicture}
    \end{center}

  \item[\quotemode] Identifies code that \textcolor{quote}{\textbf{builds ASTs}} (like \compilemode-mode) that can be manipulated at compile-time (unlike \compilemode-mode), to create run-time programs. 
  \begin{center}
    \begin{tikzpicture}
    \node (example) {$\lambda x. \, \splice[(\bind{f}{(\lambda y. \equote[{\splice[(y)] + 2}])}{\bind{a}{\equote[1]}{f a}}]) +3$};
    \draw[fill = fade, draw=fade, anchor = north west] ($(example.south west) + (0.05cm, 0.06cm)$) rectangle ($(example.south west) + (0.7cm, -0cm)$);
    
    \draw[fill = fade, draw=fade, anchor = north west] ($(example.south west) + (1.2cm, 0.06cm)$) rectangle ($(example.south west) + (3.3cm, -0cm)$);
    
    \draw[fill = fade, draw=fade, anchor = north west] ($(example.south west) + (4.08cm, 0.06cm)$) rectangle ($(example.south west) + (4.28cm, -0cm)$);
    
    \draw[fill = quote, draw=quote, anchor = north west] ($(example.south west) + (4.58cm, 0.06cm)$) rectangle ($(example.south west) + (5.15cm, -0cm)$);
    
    \draw[fill = fade, draw=fade, anchor = north west] ($(example.south west) + (5.75cm, 0.06cm)$) rectangle ($(example.south west) + (7.55cm, -0cm)$);
    
    \draw[fill = quote, draw=quote, anchor = north west] ($(example.south west) + (7.98cm, 0.06cm)$) rectangle ($(example.south west) + (8.17cm, -0cm)$);
    
    \draw[fill = fade, draw=fade, anchor = north west] ($(example.south west) + (8.6cm, 0.06cm)$) rectangle ($(example.south west) + (9.55cm, -0cm)$);
    
    \draw[fill = fade, draw=fade, anchor = north west] ($(example.south west) + (9.8cm, 0.06cm)$) rectangle ($(example.south west) + (10.5cm, -0cm)$);
    
    \end{tikzpicture}
    \end{center}
  \end{enumerate}

We may also describe how \textit{transitions} between modes occur:
\begin{enumerate}
  \item Top-level splices ($\splice$) transition from \compilemode{} (outside the splice) to \splicemode{} (within).
  \item Quotes ($\equote[e]$) transition from \splicemode{} (outside the quote) to \quotemode{} (within).
  \item Splices ($\splice$) transition from \quotemode{} (outside the splice) to \splicemode{} (within).
\end{enumerate}
Transitions between modes are illustrated in \Cref{fig:compiler-mode-transitions}.

\begin{figure}
  \centering
\begin{tikzpicture}[->,>=stealth']
  \node (c) {\Large\textbf{\textsf{\textcolor{compile}{c}}}}; 
  \node (s) at ($(c.east) + (2cm, 0cm)$) {\Large\textbf{\textsf{\textcolor{splice}{s}}}};
  \node (q) at ($(s.south) + (0cm, 2cm)$) {\Large\textbf{\textsf{\textcolor{quote}{q}}}};
  
  \draw (c) to node[anchor=north, font=\footnotesize, align=center]{$\splice$ \\ (top-level)} (s);

  \draw (s.north east) to[out=40, in=-40] node[anchor=west, font=\footnotesize]{$\equote$} (q.south east);
  \draw (q.south west) to[out=-140, in=140] node[anchor=east, font=\footnotesize]{$\splice$} (s.north west);

\end{tikzpicture}
\caption{Transitions between modes \compilemode{}, \splicemode{}, and \quotemode{}. Top-level splices transition from \compilemode{} to \splicemode{}, quotes transition from \splicemode{} to \quotemode{}, and splices (under quotes) transition from \quotemode{} to \splicemode{}.}%
\label{fig:compiler-mode-transitions}
\end{figure}

Since \sourceLang{} has only two levels, and my type system will ban nested splices and quotations ($\splice[\splice]$ and $\equote[\equote]$ are not valid program fragments), the compiler mode will uniquely identify the level (\compilemode{} and \quotemode{} imply level $0$, and \splicemode{} level $-1$). I thus drop the level from my typing judgement, leaving it implicit.

\newcommand{\cqtypejudge}[3][\Gamma]{{#1} \vdash_{\compilemode \mid \quotemode} {#2} : {#3}}
\newcommand{\ctypejudge}[3][\Gamma]{{#1} \vdash_{\compilemode} {#2} : {#3}}
\newcommand{\qtypejudge}[3][\Gamma]{{#1} \vdash_{\quotemode} {#2} : {#3}}
\newcommand{\stypejudge}[3][\Gamma]{{#1} \vdash_{\splicemode} {#2} : {#3}}

\newcommand{\runtimecomptype}[2]{{#1}^0 \, ! \, {#2}}
\newcommand{\compiletimetype}[1]{{#1}^{-1}}
\newcommand{\compiletimecomptype}[2]{{#1}^{-1} ! {#2}}

Further, the typing judgements for \compilemode{} and \quotemode{} will be identical in almost all cases. To avoid repetition, I introduce the following notation
\[\cqtypejudge{e}{T}\]
Where, for example, the typing rule 
\[\inferrule{\cqtypejudge[\Gamma_1]{e_1}{T_1} \\ \cdots \\ \cqtypejudge[\Gamma_n]{e_n}{T_n} }{\cqtypejudge{e}{T}}\]
stands for two typing rules, one in \compilemode{}-mode and one in \quotemode{}-mode.
\begin{center}
\begin{minipage}[t]{0.3\textwidth}
  \centering 
  $\inferrule{\ctypejudge[\Gamma_1]{e_1}{T_1} \\\\ \cdots \\\\ \ctypejudge[\Gamma_n]{e_n}{T_n} }{\ctypejudge{e}{T}}$
\end{minipage}%
\begin{minipage}[t]{0.3\textwidth}
  \centering
  $\inferrule{\qtypejudge[\Gamma_1]{e_1}{T_1} \\\\ \cdots \\\\ \qtypejudge[\Gamma_n]{e_n}{T_n} }{\qtypejudge{e}{T}}$
\end{minipage}
\end{center}

The \compilemode{} and \quotemode{}-mode typing rules are summarised in \Cref{fig:source-cq-typing-rules}, and the \splicemode{}-mode typing rules are summarised in \Cref{fig:source-s-typing-rules}. The rules are extremely similar to the \efflang{} typing rules (\Cref{fig:efflang-type-system}, \Cpageref{fig:efflang-type-system}). 

\begin{figure}
\begin{source-desc}
  {\large\textbf{\compilemode{} and \quotemode{} Typing Rules}}

  \begin{center}
  \begin{minipage}[t]{0.3\textwidth}
    \centering
    $\inferrule[(Nat)]{ \\ }{\cqtypejudge{n}{\runtimecomptype{\mathbb{N}}{\Delta}}}$
  \end{minipage}%
  \begin{minipage}[t]{0.3\textwidth}
    \centering
    $\inferrule[(Var)]{\Gamma(x) = T^0}{\cqtypejudge{x}{\runtimecomptype{T}{\Delta}}}$
  \end{minipage}%
  \begin{minipage}[t]{0.4\textwidth}
    \centering
$\inferrule[(Lambda)]{\cqtypejudge[\Gamma, x: T_1^0]{e}{\runtimecomptype{T_2}{\Delta;\xi}}}{\cqtypejudge{\lambda x.e}{\runtimecomptype{(\functionType[\xi]{T_1^0}{T_2^0})}{\Delta}}}$
\end{minipage}

\vspace{5mm}

\begin{minipage}[t]{0.5\textwidth}
\centering
$\inferrule[(App)]{\cqtypejudge[\Gamma]{v_1}{\runtimecomptype{(\functionType[\xi]{T_1^0}{T_2^0})}{\Delta}} \\ \cqtypejudge[\Gamma]{v_2}{\runtimecomptype{T_1}{\Delta}}}{\cqtypejudge{v_1 v_2}{\runtimecomptype{T_2}{\Delta;\xi}}}$
\end{minipage}%
\begin{minipage}[t]{0.5\textwidth}
  \centering
  $\inferrule[(Continue)]{\cqtypejudge[\Gamma]{v_1}{\runtimecomptype{(\continuationType[\xi]{T_1^0}{T_2^0})}{\Delta}} \\ \cqtypejudge[\Gamma]{v_2}{\runtimecomptype{T_1}{\Delta}}}{\cqtypejudge{\continue{v_1}{v_2}}{\runtimecomptype{T_2}{\Delta;\xi}}}$
  \end{minipage}

  \vspace{5mm}

  \begin{minipage}[t]{0.45\textwidth}
    \centering
    $\inferrule[(Return)]{  \\\\ \cqtypejudge[\Gamma]{v}{\runtimecomptype{T}{\Delta}}}{\cqtypejudge{\return{v}}{\runtimecomptype{T}{\Delta;\xi}}}$
  \end{minipage}%
  \begin{minipage}[t]{0.45\textwidth}
    \centering
    $\inferrule[(Do)]{\cqtypejudge[\Gamma]{e_1}{\runtimecomptype{T_1}{\Delta;\xi}} \\ \cqtypejudge[\Gamma, x:T_1^0]{e_2}{\runtimecomptype{T_2}{\Delta;\xi}}}{\cqtypejudge{\bind{x}{e_1}{e_2}}{\runtimecomptype{T_2}{\Delta;\xi}}}$
  \end{minipage}

\vspace{5mm}

\begin{minipage}[t]{0.5\textwidth}
  \centering
  $\inferrule[(Op)]{  \\\\ \cqtypejudge[\Gamma]{v}{\runtimecomptype{T_1}{\Delta}} \\ \texttt{op}: T_1^0 \to T_2^0 \in \Sigma \\ \texttt{op} \in \xi}{\cqtypejudge{\op{v}}{\runtimecomptype{T_2}{\Delta;\xi}}}$
\end{minipage}%
\begin{minipage}[t]{0.5\textwidth}
  \centering
  $\inferrule[(Handle)]{\cqtypejudge[\Gamma]{e}{\runtimecomptype{T_1}{\Delta;\xi}} \\ \cqtypejudge{h}{(\handlerType{\runtimecomptype{T_1}{\xi}}{\runtimecomptype{T_2}{\xi'}})\,!\,\Delta} \\ \forall \texttt{op} \in \Delta \setminus \Delta' \, . \, \texttt{op} \in \textsf{dom}(h)}{\cqtypejudge{\handleWith{e}{h}}{\runtimecomptype{T_2}{\Delta;\xi'}}}$
\end{minipage}

\vspace{5mm}

\begin{minipage}[t]{\textwidth}
  \centering
$\inferrule[(Ret-Handler)]{\cqtypejudge[\Gamma, x: T_1^0]{e}{\runtimecomptype{T_2}{\Delta;\xi'}}}{\cqtypejudge{\returnHandler{x}{e}}{(\handlerType{\runtimecomptype{T_1}{\xi}}{\runtimecomptype{T_2}{\xi'}})\,!\,\Delta}}$
\end{minipage}

\vspace{5mm}

\begin{minipage}[t]{1\linewidth}
  \centering
$\inferrule[(Op-Handler)]{\texttt{op}: A^0 \to B^0 \in \Sigma 
\\\\ \cqtypejudge{h}{(\handlerType{\runtimecomptype{T_1}{\xi}}{\runtimecomptype{T_2}{\xi'}})\,!\,\Delta}
\\\\ \cqtypejudge[\Gamma, x: A^0, k: {(\continuationType[\xi']{B^0}{T_2^0})^0} ]{e}{\runtimecomptype{T_2}{\Delta;\xi'}} \\ \xi\ \subseteq \xi \setminus \{ \texttt{op} \} \\ \opHandler{x'}{k'}{e'} \notin h} {\cqtypejudge{h;\opHandler{x}{k}{e}}{(\handlerType{\runtimecomptype{T_1}{\xi}}{\runtimecomptype{T_2}{\xi'}})\,!\,\Delta}}$
\end{minipage}

\vspace{5mm}

\begin{minipage}[t]{0.5\textwidth}
  \centering
  $\inferrule[(\compilemode{}-Splice)]{\stypejudge[\Gamma]{e}{\textsf{Code}(T^0 \, ! \, \xi)^{-1} \, ! \, \Delta}}{\ctypejudge{\splice}{\runtimecomptype{T}{\Delta ; \xi}}}$
\end{minipage}%
\begin{minipage}[t]{0.5\textwidth}
  \centering
  $\inferrule[(\quotemode{}-Splice)]{\stypejudge[\Gamma]{e}{\textsf{Code}(T^0 \, ! \, \xi)^{-1} \, ! \, \Delta}}{\ctypejudge{\splice}{\runtimecomptype{T}{\Delta ; \xi}}}$
\end{minipage}

\end{center}
\end{source-desc}
\caption{The \compilemode{}-mode and \quotemode{}-mode typing rules for \sourceLang{}. The rules are nearly identical to the \efflang{} typing rules, with the exception of level annotations on types, except that everything -- including values like $1$ -- are typed as compile-time computations. Further, two additional rules, \textsc{(\compilemode{}-Splice)} and \textsc{(\quotemode{}-Splice)} formalise the transition to \splicemode{}-mode. The former is top-level splice and the latter is splice.}%
\label{fig:source-cq-typing-rules}
\end{figure}

\begin{figure}
  \begin{source-desc}
    {\large\textbf{\splicemode{} Typing Rules}}
    \begin{center} 
    \begin{minipage}[t]{0.2\textwidth}
      \centering
      $\inferrule[(\splicemode{}-Nat)]
      { \\ }
      {\stypejudge{n}{\compiletimetype{\mathbb{N}}}}$
      \end{minipage}%
  \begin{minipage}[t]{0.2\textwidth}
    \centering
  $\inferrule[(\splicemode{}-Var)]
  {\Gamma(x) = \compiletimetype{T}}
  {\stypejudge{x}{\compiletimetype{T}}}$
  \end{minipage}

  \vspace{5mm}

  \begin{minipage}[t]{0.5\textwidth}
    \centering
  $\inferrule[(\splicemode{}-Function)]
  {\stypejudge[\Gamma, x:\compiletimetype{T_1}]{c}{\effectType{\compiletimetype{T_2}}}}
  {\stypejudge{\continuation{x}{c}}{\compiletimetype{(\functionType{\compiletimetype{T_1}}{\compiletimetype{T_2}})}}}$
  \end{minipage}%
  \begin{minipage}[t]{0.5\textwidth}
  \centering
$\inferrule[(\splicemode{}-Continuation)]
  {\stypejudge[\Gamma, x:\compiletimetype{T_1}]{c}{\effectType{\compiletimetype{T_2}}}}
  {\stypejudge{\continuation{x}{c}}{\compiletimetype{(\continuationType{\compiletimetype{T_1}}{\compiletimetype{T_2}})}}}$
\end{minipage}
  
  \vspace{5mm}
  
  \begin{minipage}[t]{0.5\textwidth}
    \centering
  $\inferrule[(\splicemode{}-App)]
    {\stypejudge{v_1}{\compiletimetype{(\functionType{\compiletimetype{T_1}}{\compiletimetype{T_2}})}} \\ \stypejudge{v_2}{\compiletimetype{T_1}}}
    {\stypejudge{v_1 \, v_2}{\effectType{\compiletimetype{T_2}}}}$
  \end{minipage}%
  \begin{minipage}[t]{0.5\textwidth}
    \centering
  $\inferrule[(\splicemode{}-Continue)]
    {\stypejudge{v_1}{\compiletimetype{(\continuationType{\compiletimetype{T_1}}{\compiletimetype{T_2}})}} \\ \stypejudge{v_2}{\compiletimetype{T_1}}}
    {\stypejudge{\continue{v_1}{v_2}}{\effectType{\compiletimetype{T_2}}}}$
  \end{minipage}

  \vspace{5mm}

  \begin{minipage}[t]{0.45\textwidth}
    \centering
  $\inferrule[(\splicemode{}-Return)]
    {  \\\\ \stypejudge{v}{\compiletimetype{T}}}
    {\stypejudge{\return{v}}{\effectType{\compiletimetype{T}}}}$
  \end{minipage}%
  \begin{minipage}[t]{0.45\textwidth}
    \centering
  $\inferrule[(\splicemode{}-Do)]
    {\stypejudge{c_1}{\effectType{\compiletimetype{T_1}}} \\ \stypejudge[\Gamma, x: T_1]{c_2}{\effectType{\compiletimetype{T_2}}}}
    {\stypejudge{\bind{x}{c_1}{c_2}}{\effectType{\compiletimetype{T_2}}}}$
  \end{minipage}
  
  \vspace{5mm}
  % \begin{minipage}[t]{0.5\textwidth}
  %   \centering
  % $\inferrule[(Do)]
  %   {\stypejudge{c_1}{\effectType{T_1}} \\ \stypejudge[, x: T_1]{c_2}{\effectType{T_2}}}
  %   {\stypejudge{\bind{x}{c_1}{c_2}}{\effectType{T_2}}}$
  % \end{minipage}%
  \begin{minipage}[t]{0.5\textwidth}
    \centering
  $\inferrule[(\splicemode{}-Op)]
    {  \\\\ \stypejudge{v}{\compiletimetype{T_1}} \\ \texttt{op}: \compiletimetype{T_1} \rightarrow \compiletimetype{T_2} \in \Sigma \\ \texttt{op} \in \Delta}
    {\stypejudge{\op{v}}{\effectType{\compiletimetype{T_2}}}}$
  \end{minipage}%
  \begin{minipage}[t]{0.5\textwidth}
    \centering
  $\inferrule[(\splicemode{}-Handle)]
    {\stypejudge{c}{\effectType{\compiletimetype{T_1}}} \\ \stypejudge{h}{\compiletimetype{(\handlerType{\effectType{\compiletimetype{T_1}}}{\effectType[\Delta']{\compiletimetype{T_2}}})}} \\ \forall \textsf{op} \in \Delta \setminus \Delta'. \, \textsf{op} \in \textsf{dom}(h)}
    {\stypejudge{\handleWith{c}{h}}{\effectType[\Delta']{\compiletimetype{T_2}}}}$
  \end{minipage}\\

  \vspace{5mm}

  \begin{minipage}[t]{\textwidth}
    \centering
  $\inferrule[(\splicemode{}-Ret-Handler)]
    {\stypejudge[\Gamma, x:\compiletimetype{T_1}]{c}{\effectType[\Delta']{\compiletimetype{T_2}}}}
    {\stypejudge{\returnHandler{x}{c}}{\compiletimetype{(\handlerType{\effectType{\compiletimetype{T_1}}}{\effectType[\Delta']{\compiletimetype{T_2}}})}}}$
  \end{minipage}
  
  \vspace{5mm}
  % \[\inferrule [(Lam)]
  % {\stypejudge[, x:T_1]{t}{\sEffect{T_2}}}
  % {\stypejudge{\function{x}{t}}{\sFunctionType{T_1}{T_2}}}\]
  
  % \[\inferrule[(App)]
  % {\stypejudge{n_1}{\sFunctionType{T_1}{T_2}} \\
  \begin{minipage}[t]{1\linewidth}
    \centering
  $\inferrule[(\splicemode{}-Op-Handler)]
    { \texttt{op}: \compiletimetype{A} \to \compiletimetype{B} \in \Sigma \\\\ 
      \stypejudge{h}{\compiletimetype{(\handlerType{\effectType{\compiletimetype{T_1}}}{\effectType[\Delta']{\compiletimetype{T_2}}})}}\\
      \stypejudge[\Gamma, x:\compiletimetype{A}, k:{\compiletimetype{(\continuationType[\Delta']{\compiletimetype{B}}{\compiletimetype{T_2}})}} ]{c}{\effectType[\Delta']{\compiletimetype{T_2}}}\\
      \Delta' \subseteq \Delta \setminus \{ \texttt{op} \} \\
             \opHandler{x'}{k'}{c'} \notin h}
    {\stypejudge{h ; \opHandler{x}{k}{c}}{\compiletimetype{(\handlerType{\effectType{\compiletimetype{T_1}}}{\effectType[\Delta']{\compiletimetype{T_2}}})}}}$
  \end{minipage}

  \vspace{5mm}

  \begin{minipage}[t]{\linewidth}
    \centering
    $\inferrule[(Quote)]{\qtypejudge{e}{\runtimecomptype{T}{\Delta ; \xi}}}{\stypejudge{\equote}{\effectType{\compiletimetype{\textsf{Code}(\runtimecomptype{T}{\xi})}}}}$
  \end{minipage}

\end{center}
  \end{source-desc}
\caption{The \splicemode{}-mode typing rules for \sourceLang{}. The rules are exactly identical to the \efflang{} typing rules, with the exception of level annotations on types, and the additional \textsc{(Quote)} rule, which makes level $0$ code available at compile-time. }%
\label{fig:source-s-typing-rules}
\end{figure}

\section{The core language: \texorpdfstring{\coreLang{}}{Lambda-Op-AST}}\label{section:core-lang}
I now describe the core language, \coreLang{}, which is an extremely simple extension of \efflang{}. For clarity, I rename \efflang{} values ($v$) and compuations ($c$) to \coreLang{} normal forms $n$ and terms $t$ respectively.  

To construct \coreLang{}, I extend \efflang{} in two ways. First, I add machinery for metaprogramming:



\begin{enumerate}
\item AST nodes (like \texttt{Nat})
\item $\gensym{}$, a primitive for generating fresh binders of type $T$ (\texttt{Var} nodes)
\item $\varToAST{}$, primitive for turning binders into \textit{usages} of binders
\end{enumerate}

Second, I extend \efflang{} with machinery for scope extrusion checking. This comprises:
\begin{enumerate}
\item \err{}, an error state for indicating the presence of scope extrusion,
\item \textbf{\texttt{check}} and \textbf{\texttt{check$_M$}}, two different checks for scope extrusion (whose semantics I will later expand on),
\item \textbf{\texttt{dlet}}, a primitive for tracking scopes when building ASTs 
\item \textbf{\texttt{tls}}, a marker for where top-level splices would have occurred in the source program. This is for \textit{ease of reasoning only}. 
\end{enumerate}
\[
\begin{array}{ll}
  n ::= & \ldots \mid \Nat{m} \mid \Var{\alpha}{T} \mid \Lam{n_1}{n_2} \mid \App{n_1}{n_2} \mid \Continue{n_1}{n_2}\\ 
  &\mid \Ret{n} \mid \Do{n_1}{n_2}{n_3} \mid \Op{n} \mid \Hwith{n_1}{n_2} \\ 
  & \mid \Hret{n_1}{n_2}(n_1, n_2) \mid \Hop{n_1}{n_2}{n_3}{n_4} \\
  &\mid \err\\
  t ::= & \ldots \mid \checkfv{n} \mid \checkm{n} \mid \dlet{n}{t} \mid \tls{t}
\end{array}
 \]

 \subsection{Type System}

 \subsection{Operational Semantics}
 
 
\section{Elaboration from \texorpdfstring{\sourceLang{}}{Lambda-Op-Quote-Splice} to \texorpdfstring{\coreLang{}}{Lambda-Op-AST}}\label{section:elaboration}

\section{Metatheory}\label{section:metatheory}