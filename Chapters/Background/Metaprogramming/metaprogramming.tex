\section{Metaprogramming}\label{section:metaprogramming-technical}
What is MacoCaml-style metaprogramming? I will provide an answer in two steps. First, I motivate metaprogramming, by illustrating the challenge of writing code that is both fast and maintainable (\Cref{subsection:metaprogramming-motivation}). Second, I will consider the design space of metaprogramming (\Cref{subsection:metaprogramming-design}), highlighting decisions made by MacoCaml.

\subsection{Metaprogramming for Fast and Maintainable Code}\label{subsection:metaprogramming-motivation}

Metaprogramming helps programmers write fast and maintainable code. How does one write fast and maintainable code? A naïve answer is ``by being a skilled programmer''\footnote{Though not the worst answer: ``use ChatGPT''}. Programmer skill is insufficient, because maintainability and efficiency are in constant tension. 

I illustrate this tension by considering a concrete problem. Consider computing the gradient of a differentiable function as part of backpropagation over a neural network. More precisely, assume $f$ of type \mintinline{ocaml}{differentiable}

\begin{ocaml}
type differentiable = Sin | Tanh | Sigmoid | `$\ldots$`
                    | Polynomial of float list
                    | Compose of differentiable * differentiable
\end{ocaml}
For example, the following expression represents $\sin\circ \tanh$. 
% or $\cos \circ \tanh(x) $ depending on if the argument is \mintinline{ocaml}{true} or \mintinline{ocaml}{false}. 
\begin{ocaml}
Compose(Tanh, Sin)
\end{ocaml}

We wish to write a function \mintinline{ocaml}{grad} such that $\text{\mintinline{ocaml}{grad}} \, f = f'$. For simplicity, assume the existence of a helper function, \mintinline{ocaml}{grad_of}, that returns the gradient of basic functions. For example, \mintinline{ocaml}{grad_of Sin 0.0} $=$ \mintinline{ocaml}{cos 0.0} $=$ \mintinline{ocaml}{1.0}. \mintinline{ocaml}{grad_main} (\Cref{listing:ocaml-grad-main}) illustrates one way to compute gradients maintainably: 

\begin{code}
\begin{ocamllst}
let rec grad_main f x = match f with
  | Sin
  | Tanh
  | Sigmoid
  | `$\ldots$`
  | Polynomial(_) -> grad_of f x 
  | Compose(f, g) -> (grad_main f x) * (grad_main g (app f x))
\end{ocamllst}
\captionof{listing}{A maintainable implementation of \mintinline{ocaml}{grad}}
\label{listing:ocaml-grad-main}
\end{code}

However, \mintinline{ocaml}{grad_main} may not be the most efficient implementation. Performing a \mintinline{ocaml}{match} on every recursive call might result in expensive branches. If \mintinline{ocaml}{x} is a vector, and the weights of a polynomial are vectors, then \mintinline{ocaml}{grad_main} could hide opportunities for cache prefetching.

If $f$ is known in advance, for example, $f =$ \text{\mintinline{ocaml}{Compose(Tanh, Sin)}}, we could implement a more efficient \mintinline{ocaml}{grad_fast} function (\Cref{listing:ocaml-grad-fast}), whose body is simply a hardcoded equation:

\begin{code}
\begin{ocamllst}
let grad_fast x = (cos x) /. (cosh (sin x) ** 2)
\end{ocamllst}
\captionof{listing}{A fast implementation of \mintinline{ocaml}{grad}, assuming $f =$ \text{\mintinline{ocaml}{Compose(Tanh, Sin)}}}
\label{listing:ocaml-grad-fast}
\end{code}    

Although \mintinline{ocaml}{grad_fast} only works for a single $f$, it has eliminated the branching overhead, and enabled opportunities for prefetching. It is thus likely to be faster.

The \mintinline{ocaml}{grad} example illustrates the trade-off between maintainability and efficiency.\ \mintinline{ocaml}{grad_main} is maintainable in part because it parameterises over $f$. More generally, abstraction centralises implementations, thus reducing maintenance costs. However, \mintinline{ocaml}{grad_fast} is more efficient because it is more specialised to a specific $f$ (\text{\mintinline{ocaml}{Compose(Tanh, Sin)}}). More generally, many compiler optimisations, like monomorphisation, eliminate abstraction, simplifying functions by applying known arguments in advance. 

The tension between maintainability (abstraction) and efficiency (specialisation) has also been observed in regex matching \citep{tratt-2008}, parsing \citep{yallop-2023}, linking \citep{servetto-2013}, statistical modelling \citep{wickham-2019}, and hardware design \citep{vandebon-2021}.

A more informed answer might therefore be ``by letting the compiler generate optimised versions of my maintainable code''. Not quite: for reasons both theoretical and practical, compiler optimisations can be insufficient. In theory, we proposed an optimisation that assumed we would always know $f$ at, or before, compile-time. Is this a reasonable assumption? It is: we assumed that \mintinline{ocaml}{grad} would perform backpropagation over neural networks. The network over which backpropagation is performed is known at compile-time. However, notice that this justification appeals to domain-specific knowledge regarding how \mintinline{ocaml}{grad} will be used. In the general case, \mintinline{ocaml}{grad} could be applied to a function not known until runtime. It is not feasible to expect a compiler to spot all opportunities for optimisation \citep{rice-53}. In practice, while compiler engineers might have an economic incentive to write optimisations for the machine learning community, this may not be true for less lucrative domains \citep{robinson-01}. Even in machine learning, many libraries are built on top of existing languages, like \texttt{Python}, which might not perform the desired optimisations.

How does one write maintainable and efficient code, \textbf{when one cannot trust the compiler to optimise one's code}?

One answer is metaprogramming, which gives users the ability to perform code-generation. Programmers may thus take matters into their own hands: manually generating optimised code when the compiler may not automatically do so for them. The \mintinline{ocaml}{grad} function in \texttt{JAX}, a Python-based machine learning framework, uses metaprogramming for precisely this purpose \citep{jax-grad-metaprogramming}.

\subsubsection{Speeding up exponentiation with Metaprogramming}
Metaprogramming allows for code-generation via the creation and manipulation of abstract syntax trees (ASTs). I will now illustrate how metaprogramming works with reference to \texttt{MacoCaml}, which implements \textit{compile-time} metaprogramming. While the \mintinline{ocaml}{grad} example motivated metaprogramming, for pedagogical reasons, I switch to a morally equivalent, but simpler example: raising an integer \mintinline{ocaml}{x} to an exponent \mintinline{ocaml}{n}. One maintainable implementation is the \mintinline{ocaml}{pow} function (\Cref{listing:ocaml-pow-maintainable}):

\begin{code}
\begin{ocamllst}
let rec pow (n: int) (x: int) = 
  if n == 0 then 1 
  else x * pow (n-1) x
\end{ocamllst}
\captionof{listing}{A maintainable implementation of an exponentiation function}
\label{listing:ocaml-pow-maintainable}
\end{code}

\mintinline{ocaml}{pow}, which can be applied to any exponent \mintinline{ocaml}{n}, is analogous to \mintinline{ocaml}{grad_main} (\Cref{listing:ocaml-grad-main}), which could be applied to \textit{any} differentiable function $f$. 

However, should we know the exponent in advance, for example \mintinline{ocaml}{n} $ =$ \mintinline{ocaml}{2}, then a more efficient, but less maintainable implementation, is the \mintinline{ocaml}{square} function (\Cref{listing:ocaml-pow-fast}) 

\begin{code}
\begin{ocamllst}
let square x = x * x
\end{ocamllst}
\captionof{listing}{An efficient implementation of exponentiation, assuming \mintinline{ocaml}{n} $=$ \mintinline{ocaml}{2}}
\label{listing:ocaml-pow-fast}
\end{code}

\noindent \mintinline{ocaml}{square} is analogous to \mintinline{ocaml}{grad_fast} (\Cref{listing:ocaml-grad-fast}).

Metaprogramming can be utilised to write a function, \mintinline{ocaml}{pow_gen}, which resembles \mintinline{ocaml}{pow} (inheriting its maintainability), but that generates a program which resembles \mintinline{ocaml}{square} (inheriting its efficiency). \Cref{listing:ocaml-pow-gen} presents the meta-programmed \mintinline{ocaml}{pow_gen} function. Compilation generates the body of \mintinline{ocaml}{square y} (line 4), \mintinline{ocaml}{y * y * 1}. I will now explain the mechanics of generation.

\begin{code}
\begin{macocamllst}
macro rec pow_gen (n: int) (x: int expr) = 
  if n == 0 then <<1>> 
  else <<$x * $(pow_gen (n-1) x)>>
let square y = $(pow_gen 2 <<y>>) (*after compile-time: y * y * 1 *)
square 3 (*at runtime: 9*)
\end{macocamllst}
\captionof{listing}{A meta-programmed \mintinline{ocaml}{pow_gen} function, which resembles \mintinline{ocaml}{pow} but generates \mintinline{ocaml}{square}}
\label{listing:ocaml-pow-gen}
\end{code}

% Notice that \mintinline{ocaml}{pow_gen} broadly resembles \mintinline{ocaml}{pow}, with the exception of \textit{metaprogramming annotations} like \mintinline{ocaml}{<<>>} and \mintinline{ocaml}{$}. These . 
Recall that (compile-time) metaprogramming gives the programmer the ability to generate programs at compile-time, for use at run-time. We may build this in two steps, by:
\begin{enumerate}
  \item Deciding on a representation for code values, such that code can be created, and manipulated by programs. Once we have a representation for code values, it is possible to write expressions that return code values. These expressions serve as program generators.
  \item Building a mechanism for executing expressions \textit{at compile-time}. We can constrain this mechanism, using types, so only generators can be executed at compile-time. 
\end{enumerate}

First, we represent code values as ASTs. Generated programs are ASTs, and program generators are expressions that evaluate to ASTs. We can assume, for clarity, that the language offers, for each program construct, a corresponding AST node -- for example, the integer \mintinline{ocaml}{1} has AST node \mintinline{ocaml}{Int(1)}. If a program has type \mintinline{ocaml}{'a}, then its AST node has type \mintinline{ocaml}{'a expr}. One can now write program generators, that evaluate to ASTs, for example:  
\begin{macocaml}
let rec pow_gen (n: int) (x: int expr) = 
  if n == 0 then Int(1) 
  else Mul(x, (pow_gen (n-1) x))
pow_gen 2 Int(3) (*Mul(Int(3), Mul(Int(3), 1))*)
pow_gen 2 Var(y) (*Mul(Var(y), Mul(Var(y), 1))*)
pow_gen 3 Var(y) (*Mul(Var(y), Mul(Var(y), Mul(Var(y), 1)))*)
\end{macocaml}

Second, we need a mechanism to execute expressions at compile-time. In \texttt{MacoCaml}, this is the ``top-level splice'', a splice (\mintinline{ocaml}{$}) annotation not surrounded by quotes (\mintinline{ocaml}{<<>>}). For example, in \Cref{listing:ocaml-pow-gen}, there is only one top-level splice, on line 4: \mintinline{ocaml}{$(pow_gen 2 <<y>>)}. We may now shift program generators (and only program generators) under top-level splices, to perform generation at compile time. Note that to access \mintinline{ocaml}{pow_gen} at compile-time, we must also move it under the top-level splice. 

\begin{macocaml}
let square y = $(let rec pow_gen (n: int) (x: int expr) = ...
                 in pow_gen 2 Var(y)) 
               (*Mul(Var(y), Mul(Var(y), 1))*)
let cube y   = $(let rec pow_gen (n: int) (x: int expr) = ...
                 in pow_gen 3 Var(y)) 
               (*Mul(Var(y), Mul(Var(y), Mul(Var(y), 1)))*)
\end{macocaml}

To allow compile-time functions, like \mintinline{ocaml}{pow_gen}, to be re-used across multiple top-level splices, \texttt{MacoCaml} introduces the \mintinline{\macocamlLexer}{macro} (\Cref{listing:ocaml-macros})

\begin{code}
\begin{macocamllst}
macro rec pow_gen (n: int) (x: int expr) = 
  if n == 0 then Int(1) 
  else Mul(x, (pow_gen (n-1) x))
let square y = $(pow_gen 2 Var(y)) (*Mul(Var(y), Mul(Var(y), 1))*)
let cube y = $(pow_gen 3 Var(y)) (*Mul(Var(y), Mul(Var(y), Mul(Var(y), 1)))*)
  \end{macocamllst}

  \captionof{listing}{In \texttt{MacoCaml}, \mintinline{\macocamlLexer}{macro} allows for definitions to be shared across top-level splices}

  \label{listing:ocaml-macros}
\end{code}
% To allow definitions to

% The ability to create and manipulate ASTs, combined with the ability to construct and execute functions at compile-time, allows for compile-time code generation. In the example below, we execute \mintinline{ocaml}{add_zero} at compile-time to generate the program \mintinline{ocaml}{(1*2)+0}.
Further, rather than explicit AST constructors, ASTs are created by the \mintinline{ocaml}{<<>>} (``quote'') and \mintinline{ocaml}{$} annotations. Quotation creates ASTs, by converting a program into its AST representation. For example,
\[\text{\mintinline{ocaml}{<< $x + 0 >>}} = \text{\mintinline{ocaml}{Plus(Var(x), Int(0))}}\]
Under a quotation, the \mintinline{ocaml}{$} annotation stops this conversion, allowing for programs that \textit{manipulate} ASTs. 

\[ \text{\mintinline{ocaml}{<< $x + 0 >>}} = \text{\mintinline{ocaml}{Plus(x, Int(0))}} \]
In \texttt{MacoCaml}, the programmer interleaves quotes and splices to perform code generation
\[\text{\mintinline{ocaml}{<< $(add_zero <<1>>) + 0 >>}} = \text{\mintinline{ocaml}{Plus(add_zero Int(1), Int(0))}} \]
Notice that \mintinline{ocaml}{$} is overloaded. We must be careful to disambiguate between ``top-level splices'', which execute programs at compile-time, and splices under quotations, which stop conversion to AST. 

Re-writing \Cref{listing:ocaml-macros} in this style (being careful about non-top-level splices), we obtain exactly \Cref{listing:ocaml-pow-gen}.

Applying this technique to the \mintinline{ocaml}{grad} example, we obtain
\begin{macocaml}
macro rec grad_gen f x = match f with
  | Sin
  | Tanh
  | Sigmoid
  | ...
  | Polynomial(_) -> grad_of f x 
  | Compose(f, g) -> <<$(grad_gen f x) * $(grad_gen g (app f x))>>

let grad_fast x = $(grad_gen Compose(Tanh, Sin) <<x>>)
\end{macocaml}
whathere \mintinline{ocaml}{grad_of} and \mintinline{ocaml}{app} are appropriately modified.
\subsection{The Design Space of Metalanguages}\label{subsection:metaprogramming-design}
Different metalanguages provide slightly different variants of metaprogramming to the user. In this section, I broadly taxonomise these languages by considering three key design decisions:
\begin{enumerate}
  \item \textbf{\textsf{Homogenous or Heterogenous}}\\
         \textit{Do the generated (``object'') and generating (``meta'') languages agree or differ?}
  
        If the object and meta languages are the same, this is known as {homogenous} metaprogramming. Otherwise, it is {heterogenous} \citep{kiselyov-2024}.

        \texttt{MacoCaml} allows for homogenous metaprogramming, where \texttt{OCaml} code generates \texttt{OCaml} code. In contrast, \texttt{MetaHaskell} \citep{mainland-2012} programs generate \texttt{C} code, allowing for heterogenous metaprogramming. 

  \item \textbf{\textsf{Run-time or Compile-Time}} \\
        \textit{When does the generation take place?}

        Code generation could take place at compile-time (as with \texttt{MacoCaml} programs or \texttt{C} macros), or at run-time (as with \texttt{MetaOCaml} \citep{kiselyov-14}). 

        % Run-time code generation may still be used to write fast and maintainable code, by adding a phase after the traditional compilation pipeline where the compiled code is executed. 
        
        Run-time and compile-time metaprogramming differ non-trivially. Run-time metaprogramming requires a language construct (\mintinline{ocaml}{!}, or ``run'') for explicit invocation of the compiler. Further, in run-time metaprogramming, generated and generating programs may share a heap. 

        \texttt{MacoCaml} supports compile-time metaprogramming, and we will pay no further attention to run-time metaprogramming.
        
  \item \textbf{\textsf{Two-stage or Multi-stage}} \\
  \textit{How many stages of code generation are allowed?}

  When introducing \texttt{MacoCaml}, I illustrated how one uses top-level splices to perform shift computation from run-time (``level 0'') to compile-time (``level -1''). Might it be possible to shift computation from compile-time to a pre-compile-time (``level -2'' phase), for example, via a nested splice?
  \begin{macocaml}
$($ pow_gen 2 Var(y))
  \end{macocaml}
  In a two-stage system, one is restricted to operating between two levels, so this is disallowed. In contrast, in a multi-stage system, one can operate between any number of levels. Multi-stage metaprogramming is thus strictly more general than two-stage metaprogramming.
  
  Although nested splices are disallowed in \texttt{MacoCaml}, it is a multi-stage system, since entire modules may be imported at a decremented level \citep{xie-2023}. 
\end{enumerate}
  
\texttt{MacoCaml} offers homogenous, compile-time, multi-stage metaprogramming. The scope of this dissertation is slightly more restrictive: I focus on two-stage, not multi-stage metaprogramming. This restriction was motivated by a cost-benefit analysis:
\begin{enumerate}
  \item \textbf{Cost}: Since in \texttt{MacoCaml}, the module system is the only mechanism for achieving multi-stage programming, investigating multi-stage metaprogramming would require the investigation of module systems, effects, and metaprogramming. The interaction between module systems and metaprogramming is still an ongoing area of research \citep{chiang-2024}.
  \item \textbf{Benefit}: In practice, ``almost all uses'' of multi-stage metaprogramming only use two stages \citep{inoue-2012}. Further, scope extrusion can be observed, and is often studied, in two-stage systems \citep{isoda-24,kiselyov-16}.
\end{enumerate}