\section{Scope Extrusion}\label{section:scope-extrusion-technical}
I now turn my attention to scope extrusion, which arises from the unexpected interaction of effects and metaprogramming. To illustrate scope extrusion, I will first extend \efflang{} (\Cpageref{subsection:effect-handler-calculus}) with AST constructors $\Var{x}{T}$, $\texttt{Nat}(n)$, $\texttt{Lam}$, and $\texttt{Plus}$. For example, we may generate the AST of $\lambda x: \mathbb{N}. \, x + 0$ as follows:

\begin{eff}
$\return{\Lam{\Var{x}{\mathbb{N}}}{\texttt{Plus}(\Var{x}{\mathbb{N}}, \texttt{Nat}(0))}}$
\end{eff}

\Cref{listing:efflang-scope-extrusion} illustrates the problem of scope extrusion. The program constructs the AST of $\lambda x: \mathbb{N}. \, x$, but additionally performs an effect, \textbf{\texttt{extrude}}, with type $\mathbb{N} \,  \texttt{expr} \to \mathbb{N} \, \texttt{expr}$. The handler for \textbf{\texttt{extrude}} discards the continuation, simply returning the value it was given: $\Var{x}{\mathbb{N}}$. The entire program evaluates to $\Var{x}{\mathbb{N}}$, and the generated AST is ill-scoped. We say that the result of evaluation demonstrates scope extrusion.

\begin{code}
  \begin{efflst}
    $\begin{array}{l}
      \textbf{\texttt{handle}} \\
      \quad \bind{\texttt{body}}{\textbf{\texttt{extrude}}({\Var{x}{\mathbb{N}}})}{\return{\Lam{\Var{x}{\mathbb{N}}}{\texttt{body}}}} \\
      \textbf{\texttt{with}} \\
      \quad \{ \textbf{\texttt{return}}(u) \mapsto {\return{\texttt{Nat}(0)}}; \\
      \quad \; \, \textbf{\texttt{extrude}}(y, k) \mapsto {\return{y}}\}\\
    \end{array}$
  
    \vspace{2mm} 
  \textcolor{effComment}{\hrule height 0.2mm \relax}
  \vspace{2mm} 
  
  \textcolor{effComment}{$\begin{array}{l}\return{\Var{x}{\mathbb{N}}}\end{array}$}
  
  \end{efflst}
  \captionof{listing}{A \efflang{} program that evaluates to the $\Var{x}{\mathbb{N}}$. The AST is ill-scoped, and thus exhibits scope extrusion. It will be used as a running example.}%
  \label{listing:efflang-scope-extrusion}
  \end{code}

It is difficult to give a precise definition to scope extrusion, because there are multiple competing definitions \citep{kiselyov-14,kiselyov-16}, and many are given informally. For example, is scope extrusion a property of the \textit{result} of evaluation \citep{kiselyov-16}, as in \Cref{listing:efflang-scope-extrusion}, or is it a property of \textit{intermediate} configurations \citep{kiselyov-14}? We can, for example, build ASTs with extruded variables, that are bound at some future point. In \Cref{listing:efflang-maybe-scope-extrusion}, we produce the intermediate AST $\texttt{Plus}(\texttt{Nat}({0}), \Var{x}{\mathbb{N}})$, which is not well scoped. However, the result of evaluation is well scoped: $\Lam{\Var{x}{\mathbb{N}}}{\texttt{Plus}(\texttt{Nat}({0}), \Var{x}{\mathbb{N}})}$. Does \Cref{listing:efflang-maybe-scope-extrusion} exhibit scope extrusion?


\begin{code}
  \begin{efflst}
    $\begin{array}{l}
      \textbf{\texttt{handle}} \\
      \quad \bind{\texttt{body}}{\textbf{\texttt{extrude}}({\Var{x}{\mathbb{N}}}); \return{\Var{x}{\mathbb{N}}}}{\return{\Lam{\Var{x}{\mathbb{N}}}{\texttt{body}}}} \\
      \textbf{\texttt{with}} \\
      \quad \{ \textbf{\texttt{return}}(u) \mapsto \return{u}; \\
      \quad \; \, \textbf{\texttt{extrude}}(y, k) \mapsto { \bind{z}{\return{\texttt{Plus}({\texttt{Nat}({0})}, y)}}{\continue{k}{z}}} \}\\
    \end{array}$

    \vspace{2mm} 
\textcolor{effComment}{\hrule height 0.2mm \relax}
\vspace{2mm} 

\textcolor{effComment}{$\begin{array}{l}\return{\Lam{\Var{x}{\mathbb{N}}}{\texttt{Plus}(\texttt{Nat}({0}), \Var{x}{\mathbb{N}})}}\end{array}$}

\end{efflst}
\captionof{listing}{A \efflang{} program that may, or may not demonstrate scope extrusion, depending on one's definition. The final result of the program is well-scoped, but not all intermediate results are well-scoped.}%
\label{listing:efflang-maybe-scope-extrusion}
\end{code}

Nevertheless, all definitions agree on the example in \Cref{listing:efflang-scope-extrusion}. Making precise the competing definitions of scope extrusion these competing definitions, and their relation to one another, is a contribution of this dissertation. 

\subsection{Existing Solutions to the Scope Extrusion Problem}
There are multiple solutions to the problem of scope extrusion. The solution space can be broadly divided into two types of approaches: static (type-based) and dynamic. I will now survey two dynamic approaches, which I term the lazy and eager checks, and one static approach, the method of refined environment classifiers \citep{kiselyov-16,isoda-24}. 

\subsubsection{Lazy Dynamic Check}\label{subsubsection:lazy-dynamic-check}
Scope extrusion, at least, of the kind in \Cref{listing:efflang-no-scope-extrusion}, may seem trivial to resolve: evaluate the program to completion, and check that the resulting AST is well-scoped \citep{kiselyov-14}. I term this the Lazy Dynamic Check. This approach, while clearly correct and maximally expressive, is not ideal for efficiency and error reporting reasons. 

To illustrate the inefficiency of this approach, consider a slight variation of \Cref{listing:efflang-scope-extrusion}, \Cref{listing:efflang-lazy-scope-extrusion-inefficient}. In \Cref{listing:efflang-lazy-scope-extrusion-inefficient}, we can, in theory, report a warning as soon scope extrusion is detected. However, waiting for the result of the program can be much more inefficient. 

In terms of error reporting, note that, in waiting for the result of execution, the lazy dynamic check loses information about \textit{which program fragment} was responsible for scope extrusion, reducing the informativeness of reported errors \citep{kiselyov-14}.  

\begin{code}
  \begin{efflst}
    $\begin{array}{l}
      \textbf{\texttt{do}} \,  x \leftarrow \textbf{\texttt{handle}} \\
      \quad \quad \quad \quad \,\, \bind{\texttt{body}}{\textbf{\texttt{extrude}}({\Var{x}{\mathbb{N}}})}{\return{\Lam{\Var{x}{\mathbb{N}}}{\texttt{body}}}} \\
      \quad \quad \quad \,\, \textbf{\texttt{with}} \\
      \quad \quad \quad \quad \, \, \{ \textbf{\texttt{return}}(u) \mapsto {\return{\texttt{Nat}(0)}}; \\
      \quad \quad \quad \quad \, \, \; \, \textbf{\texttt{extrude}}(y, k) \mapsto {\return{y}}\}\\
      \textbf{\texttt{in}} \text{ some very long program}; \return{x}
    \end{array}$

    \vspace{2mm} 
\textcolor{effComment}{\hrule height 0.2mm \relax}
\vspace{2mm} 

\textcolor{effComment}{$\begin{array}{l}\return{\Var{x}{\mathbb{N}}}\end{array}$}

\end{efflst}
\captionof{listing}{A \efflang{} program that evaluates to the $\Var{x}{\mathbb{N}}$. Executing the entire program to determine if it exhibits scope extrusion is inefficient.}%
\label{listing:efflang-lazy-scope-extrusion-inefficient}
\end{code}

\subsubsection{Eager Dynamic Check}\label{subsubsection:eager-dynamic-check}
A second dynamic check, motivated by the problems with the Lazy Dynamic Check, adopts a stricter definition of scope extrusion. During the code generation process, one inserts checks into the running program, reporting errors when one encounters  unbound free variables in intermediate ASTs. Hence, the Eager Dynamic Check would classify the program in \Cref{listing:efflang-maybe-scope-extrusion} as exhibiting scope extrusion. The Eager Dynamic Check has been adopted by BER MetaOCaml, and offers better efficiency and error reporting guarantees over the Lazy Dynamic Check \citep{kiselyov-14}.

However, the Eager Dynamic Check is not without issue. The problem relates to the 
manner in which checks are inserted, which I will make precise later. To illustrate the problem, consider \Cref{listing:efflang-maybe-not-scope-extrusion}, a slight variation of \Cref{listing:efflang-maybe-scope-extrusion} in which the program fragment $\texttt{Plus}({\texttt{Nat}({0})}, y)$ is replaced with $y$. 

\begin{code}
  \begin{efflst}
    $\begin{array}{l}
      \textbf{\texttt{handle}} \\
      \quad \bind{\texttt{body}}{\textbf{\texttt{extrude}}({\Var{x}{\mathbb{N}}}); \return{\Var{x}{\mathbb{N}}}}{\return{\Lam{\Var{x}{\mathbb{N}}}{\texttt{body}}}} \\
      \textbf{\texttt{with}} \\
      \quad \{ \textbf{\texttt{return}}(u) \mapsto \return{u}; \\
      \quad \; \, \textbf{\texttt{extrude}}(y, k) \mapsto { \bind{z}{\return{y}}{\continue{k}{z}}} \}\\
    \end{array}$

    \vspace{2mm} 
\textcolor{effComment}{\hrule height 0.2mm \relax}
\vspace{2mm} 

\textcolor{effComment}{$\begin{array}{l}\return{\Lam{\Var{x}{\mathbb{N}}}{\texttt{Plus}(\texttt{Nat}({0}), \Var{x}{\mathbb{N}})}}\end{array}$}

\end{efflst}
\captionof{listing}{A \efflang{} program that is a slight variation of \Cref{listing:efflang-maybe-scope-extrusion}, but that (unlike \Cref{listing:efflang-maybe-scope-extrusion}) passes the Eager Dynamic Check.}%
\label{listing:efflang-maybe-not-scope-extrusion}
\end{code}

While the program in \Cref{listing:efflang-maybe-not-scope-extrusion} produces an intermediate AST with extruded variables, because of the mechanism for inserting checks, it passes the Eager Dynamic Check. I assert that this behaviour is unintuitive, and exposes too much of the internals to the programmer. 

\subsubsection{Refined Environment Classifiers}\label{subsubsection:refined-environment-classifiers}
Refined Environment Classifiers are a static check that uses the type system to prevent scope extrusion. Recall that metaprogramming involves the \textit{creation} and \textit{manipulation} of ASTs. Refined environment classifiers prevent scope extrusion by checking:

\begin{enumerate}
  \item \textit{Created} ASTs are well-scoped
  \item \textit{Manipulating} ASTs preserves well-scopedness 
\end{enumerate}

I shall first ignore the ability to manipulate ASTs, and consider how to ensure \textit{created} ASTs are well-scoped. What does it mean to be well-scoped? Consider \Cref{fig:refined-enviroment-classifiers-basic}, the AST of 
\[(\lambda f. \lambda x. f x) (\lambda y. y)\]
Informally, a scope represents a set of variables that are permitted to be free. In the example, there are four scopes: one where no variables are free, one where only $f$ is free, one where $f$ and $x$ are free, and one where $y$ is free. An AST is well-scoped \textit{at a scope} if it is well-typed, and where the only free variables are those permitted by the scope. 

Refined environment classifiers make this notion precise. Each classifier represents a scope. The AST has four scopes, corresponding to four classifiers:
\begin{itemize}
\item[$\gamma_{\bot}$\,\,] The top-level, where no free variables are permitted 
\item[$\gamma_{f}$\;\,] Only $\texttt{Var}(f)$ permitted to be free
\item[$\gamma_{fx}$] Only $\texttt{Var}(f)$ and $\texttt{Var}(x)$ permitted to be free 
\item[$\gamma_{y}$\;\,] Only $\texttt{Var}(y)$ permitted to be free
\end{itemize}
As every variable binder creates a scope, we may refer to the classifier created by $\texttt{Var}(\alpha)$, $\textsf{classifier}(\texttt{Var}(\alpha))$. For example, $\gamma_{fx}= \textsf{classifier}(\texttt{Var}(x))$.

\begin{figure}
  \centering
  \begin{tikzpicture}[level 1/.style={sibling distance=20em},
                      level 2/.style={sibling distance=8em},
                      level 3/.style={sibling distance=7em},
                      level 4/.style={sibling distance=4em}]
  
    \begin{scope}[every node/.style={font=\ttfamily}]
    \node (app) {App}
    child {node (lam-1) {Lam}
      child {node (var-1) {Var($f$)}}
      child {node (lam-2) {Lam}
             child {node (var-2) {Var($x$)}}
             child {node (app-2) {App}
                    child {node (var-1-use) {Var($f$)} }
                    child {node (var-2-use) {Var($x$)} }
             }
            }
    }
    child {node (lam-3) {Lam}
    child {node (var-3) {Var($y$)}}
    child {node (var-3-use) {Var($y$)}}};
    \end{scope}
  
  
  
    \begin{scope}
      \draw[line width=0.5mm, rounded corners,envclassifier1, dashed] ($(var-3-use.south east) + (0.4cm,-3.7cm)$) rectangle ($(var-1.north west) + (-0.3cm,3.3cm)$);
      
      \draw[line width=0.5mm, rounded corners,envclassifier1, dashed] ($(var-2-use.south east) + (0.35cm,-0.4cm)$) rectangle ($(lam-2.north west) + (-2.0cm,0.2cm)$);
      \draw[line width=0.5mm, rounded corners,envclassifier1, dashed] ($(var-2-use.south east) + (0.05cm,-0.1cm)$) rectangle ($(app-2.north west) + (-1.3cm,0.1cm)$);
      \draw[line width=0.5mm, rounded corners,envclassifier1, dashed] ($(var-3-use.south east) + (0.2cm,-0.2cm)$) rectangle ($(var-3-use.north west) + (-0.2cm,0.2cm)$);
  
      % \fill[rounded corners, envclassifier1] ($(var-1.north west) + (-0.3cm,3cm)$) rectangle ($(var-1.north west) + (-0.3cm,3cm)$);
    \end{scope}
  
    \begin{scope}[every node/.style={rounded corners, fill=envclassifier2, font=\scriptsize}]
      \node (gammabot) at ($(var-3.north east) + (2cm,3.3cm)$) {$\gamma_{\bot}$};
      \node (gammaf) at ($(lam-2.north west) + (2.7cm,0.2cm)$) {$\gamma_{f}$};
      \node (gammafx) at ($(app-2.north west) + (1.3cm,0.1cm)$) {$\gamma_{fx}$};
      \node (gammafx) at ($(var-3-use.north west) + (1.2cm,0.2cm)$) {$\gamma_{fx}$};
    \end{scope}
    
  \end{tikzpicture}
  \caption{The AST of $(\lambda f. \lambda x. f x) (\lambda y.y)$, where each scope is labelled with the corresponding environment classifier.}%
  \label{fig:refined-enviroment-classifiers-basic}
  \end{figure}

With classifiers, I may now be precise about ``the variables permitted to be free (within the scope)''. As illustrated by the nesting in \Cref{fig:refined-enviroment-classifiers-basic}, scopes are related to other scopes. For example, since the scope $\gamma_{fx}$ is created within scope $\gamma_{f}$, any variable tagged with $\gamma_f$ may be safely used in the scope $\gamma_{fx}$. I say $\gamma_f$ is compatible with $\gamma_{fx}$, and write 
\[\gamma_{f} \sqsubseteq \gamma_{fx}\]
The compatibility relation ($\sqsubseteq$) is a partial order, meaning $\sqsubseteq$ is reflexive, anti-symmetric, and transitive, and identifies a smallest classifier. Reflexivity expresses that  $\texttt{Var}(\alpha)$ may be used within the scope it creates
\[\forall \gamma. \, \gamma \sqsubseteq \gamma \] 
anti-symmetric, since nesting only proceeds in one direction
\[\forall \gamma_1, \gamma_2. \, \gamma_1 \sqsubseteq \gamma_2 \land \gamma_2 \sqsubseteq \gamma_1 \implies \gamma_1 = \gamma_2 \] 
and transitive, since we should count nestings within nestings
\[\forall \gamma_1, \gamma_2, \gamma_3. \, \gamma_1 \sqsubseteq \gamma_2 \land \gamma_2 \sqsubseteq \gamma_3 \implies \gamma_1 \sqsubseteq \gamma_3\]
$\gamma_{\bot}$ acts as the least element of this partial order
\[\forall \gamma. \, \gamma_{\bot} \sqsubseteq \gamma \] 

Given a classifier $\gamma$, I may now define the variables permitted to be free in $\gamma$, written $\textsf{permitted}(\gamma)$
\[\textsf{permitted}(\gamma) \triangleq \{ \texttt{Var}(\alpha) \mid \textsf{classifier}(\texttt{Var}(\alpha)) \sqsubseteq \gamma \}\]
For example, $\textsf{permitted}(\gamma_{fx}) = \{ \texttt{Var}(f), \texttt{Var}(x)\}$ 

I say that an AST $n$ is well-scoped at type $T$ and scope $\gamma$ if it is well-typed at $T$, and all free variables in $n$ are in $\textsf{permitted}(\gamma)$. I write 
\[\Gamma \vdash^{\gamma} n : T \]
% \[ \textsf{permitted}(\texttt{Var}(\alpha)) \triangleq \{ \gamma \mid \textsf{classifier}(\texttt{Var}(\alpha)) \sqsubseteq \gamma \} \]

Ensuring that created ASTs are well-scoped is the responsibility of the type system. The key rule is the \textsc{C-Abs} rule, which, \textbf{assuming we know that we are creating an AST}, has roughly the following shape\footnote{I will revisit this assumption when introducing the type system for my calculus}: 
\[\inferrule[(C-Abs)]{\, \gamma \in \Gamma \\ (2) \, \gamma' \text{fresh} \\ (3) \, \Gamma, \gamma', \gamma \sqsubseteq \gamma', {(x: T_1)}^{\gamma'} \vdash^{\gamma'} n: T_2}{(1) \, \Gamma \vdash^{\gamma} \lambda x. n: T_1 \longrightarrow T_2}\]
The premises and conclusions have been numbered for reference, and many  technical details have been simplified for clarity. \Cref{fig:cref-typing-rule} visually depicts the typing rule. 

\begin{enumerate}
  \item[(1)] The goal of the typing rule is to ensure the function is well-scoped at type $T_1 \to T_2$ and scope $\gamma$.
  \item[(2)] Since the function introduces a new variable binder, $\texttt{Var}(x)$, one has to create a new scope. This is achieved by picking a fresh classifier $\gamma'$.
  \item[(3)] We record the following:
  \begin{enumerate}
    \item[(a)] Since $\gamma'$ is created within the scope of $\gamma$, $\gamma \sqsubseteq \gamma'$.
    \item[(b)] $\textsf{classifier}(\texttt{Var}(x)) = \gamma'$ (as a shorthand ${(x: T_1)}^{\gamma'}$)
  \end{enumerate}
  With this added knowledge, we ensure that the function body is well-scoped at type $T_2$ and $\gamma'$.
\end{enumerate}

\begin{figure}
  \centering
  \begin{tikzpicture}[level 1/.style={sibling distance=10em}]
  
    \begin{scope}[every node/.style={font=\ttfamily}]
    \node (lam) {Lam}
    child {node (binder) {Var$(x)$}}
    child {node (body) {n}};
    \end{scope}
  
  
  
    \begin{scope}
      \draw[line width=0.5mm, rounded corners,envclassifier1, dashed] ($(body.south east) + (1.2cm,-.5cm)$) rectangle ($(binder.north west) + (-0.9cm,1.8cm)$);
      
      % \draw[line width=0.5mm, rounded corners,envclassifier1, dashed] ($(var-2-use.south east) + (0.35cm,-0.4cm)$) rectangle ($(lam-2.north west) + (-2.0cm,0.2cm)$);
      % \draw[line width=0.5mm, rounded corners,envclassifier1, dashed] ($(var-2-use.south east) + (0.05cm,-0.1cm)$) rectangle ($(app-2.north west) + (-1.3cm,0.1cm)$);
      \draw[line width=0.5mm, rounded corners,envclassifier1, dashed] ($(body.south east) + (-1cm,-0.2cm)$) rectangle ($(body.north west) + (1cm,0.2cm)$);
  
      % \fill[rounded corners, envclassifier1] ($(var-1.north west) + (-0.3cm,3cm)$) rectangle ($(var-1.north west) + (-0.3cm,3cm)$);
    \end{scope}
  
    \begin{scope}[every node/.style={rounded corners, fill=envclassifier2, font=\scriptsize}]
      \node (gammabot) at ($(binder.north west) + (5.3cm,1.8cm)$) {$\gamma$};
      \node (gammaf) at ($(body.north west) + (.6cm,0.2cm)$) {$\gamma'$};
    \end{scope}
    
  \end{tikzpicture}
  \caption{Visual depiction of the \textsc{(C-Abs)} typing rule.}%
  \label{fig:cref-typing-rule}
  \end{figure}

The above example focused on \textbf{creating} ASTs, and had no compile-time executable code. I now consider how to maintain well-scopedness while \textbf{manipulating} ASTs. I consider the ``AST'' of the scope extrusion example, \Cref{listing:efflang-scope-extrusion} (\Cref{fig:classifier-ast-scope-extrusion}). Notice that in place of AST nodes, we may now have compile-time executable code that \textit{evaluate} to AST nodes. Thus, both code and AST nodes reside within scopes. We have two classifiers: $\gamma_{\bot}$ and $\gamma_{\alpha}$, with $\textsf{classifier}(\texttt{Var}(\alpha)) = \gamma_{\alpha}$.

\begin{figure}
\centering

\begin{tikzpicture}[level 1/.style={sibling distance=15em}]

  \begin{scope}[every node/.style={font=\ttfamily}]
  \node (handler) {Lam}
  child {node (var) {Var($\alpha$)}}
  child {node (body) {\textbf{extrude}(\texttt{Var}($\alpha$)); $\return(\texttt{Var}(\alpha))$}};
  \end{scope}

  \begin{scope}
    \node[anchor=east] (lam-left) at ($(handler.west) + (-0.1cm, 0cm)$) {\textbf{\texttt{handle}}};
    \node[anchor=west] (lam-right) at ($(handler.east) + (0.1cm, 0cm)$) {\texttt{\textbf{with} \{$\ldots$,\textbf{extrude}($y, k$)$\mapsto\ldots$\} }};
  \end{scope}


  \begin{scope}[on background layer]
    \draw[line width=0.5mm, rounded corners,envclassifier1, dashed] ($(body.south east) + (0.5cm,-0.4cm)$) rectangle ($(lam-left.north west) + (-2.1cm,0.4cm)$);

    \draw[line width=0.5mm, rounded corners,envclassifier1, dashed] ($(body.south east) + (0.1cm,-0.1cm)$) rectangle ($(body.north west) + (-0.2cm,0.2cm)$);

    \begin{scope}[every node/.style={rounded corners, fill=envclassifier2, font=\scriptsize}]
      \node (gammabottom) at ($(lam-left.north west) + (7.6cm,0.4cm)$) {$\gamma_{\bot}$};
      \node (gammaalpha) at ($(body.north west) + (5.8cm,0.2cm)$) {$\gamma_{\alpha}$};
    \end{scope}
    % \draw[line width=0.5mm, rounded corners,envclassifier2a] ($(var-2-use.south east) + (0.05cm,-0.1cm)$) rectangle ($(app-2.north west) + (-1.3cm,0.1cm)$);
    % \draw[line width=0.5mm, rounded corners,envclassifier1] ($(var-3-use.south east) + (0.1cm,-0.1cm)$) rectangle ($(var-3-use.north west) + (-0.1cm,0.1cm)$);

    % \fill[rounded corners, envclassifier1] ($(var-1.north west) + (-0.3cm,3cm)$) rectangle ($(var-1.north west) + (-0.3cm,3cm)$);
  \end{scope}
  
\end{tikzpicture}

\caption{The ``AST'' of the scope extrusion example, \Cref{listing:efflang-scope-extrusion}. Notice that in place of AST nodes, we may now have compile-time executable code that \textit{evaluate} to AST nodes.}%
\label{fig:classifier-ast-scope-extrusion}
\end{figure}

Key to the prevention of scope extrusion is the typing of handlers and operations, like \textbf{\texttt{extrude}}, that manipulate ASTs. Since these rules are complex, I describe them informally. The handle expression
\[\handleWith{e}{h}\]
is in scope $\gamma_{\bot}$. Therefore, for each operation handled by $h$, such as \textbf{\texttt{extrude}}, the argument to the operation must either not be an AST, or be an AST that is well-scoped at some $\gamma \sqsubseteq \gamma_{\bot}$. 
% We 
% \[\textbf{\texttt{extrude}}: \texttt{AST}(\mathbb{N})^{\gamma_{\bot}} \to 1^{\gamma_{\bot}}\]
However, $\texttt{Var}(\alpha)$ is typed at $\gamma_{\alpha}$, and clearly, $\gamma_{\alpha} \not\sqsubseteq \gamma_{\bot}$. There is thus no way to type the scope extrusion example in \Cref{listing:efflang-scope-extrusion}.

Note that the analysis was independent of the \textit{body} of the handler. Therefore, the examples in \Cref{listing:efflang-maybe-not-scope-extrusion,listing:efflang-maybe-scope-extrusion} are \textit{also} not well-typed. Perhaps somewhat surprisingly, so too is \Cref{listing:efflang-no-scope-extrusion} (which would pass both the Eager and Lazy Dynamic Checks). Refined environment classifiers statically prevent variables ($\texttt{Var}(\alpha)$) from becoming \textit{available} in program fragments ($\texttt{\textbf{op}}(y, k) \mapsto \ldots$) where, \textit{if misused}, \textit{might} result in scope extrusion. This is, of course, an over-approximation. It means that the refined environment classifiers check prevents not only both types of scope extrusion, but even more benign examples, such as that in \Cref{listing:efflang-no-scope-extrusion}.

\begin{code}
  \begin{efflst}
    $\begin{array}{l}
      \textbf{\texttt{handle}} \\
      \quad \bind{\texttt{body}}{\textbf{\texttt{extrude}}({\Var{x}{\mathbb{N}}}); \return{\Var{x}{\mathbb{N}}}}{\return{\Lam{\Var{x}{\mathbb{N}}}{\texttt{body}}}} \\
      \textbf{\texttt{with}} \\
      \quad \{ \textbf{\texttt{return}}(u) \mapsto \return{\texttt{App}(u, \texttt{Nat}(1))}; \\
      \quad \; \, \textbf{\texttt{extrude}}(y, k) \mapsto { \return{\texttt{Nat}(0)}} \}\\
    \end{array}$

    \vspace{2mm} 
\textcolor{effComment}{\hrule height 0.2mm \relax}
\vspace{2mm} 

\textcolor{effComment}{$\begin{array}{l}\return{\texttt{Nat}(0)}\end{array}$}

\end{efflst}
\captionof{listing}{A \efflang{} program that passes the Eager and Lazy Dynamic Checks, but is not well-typed under the Refined Environment Classifiers type system.}%
\label{listing:efflang-no-scope-extrusion}
\end{code}

\Cref{listing:efflang-no-scope-extrusion} thus illustrates one of the key drawbacks of Refined Environment Classifiers: the check is too stringent, and restricts expressiveness. 